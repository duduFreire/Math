\documentclass[leqno]{article}
\usepackage[utf8]{inputenc}
\usepackage[english]{babel}
\usepackage{amsmath}
\usepackage{amsthm}
\usepackage{mathtools}
\usepackage{amsfonts}
\usepackage{amssymb}
\usepackage{mathabx}
\usepackage{yfonts}
\usepackage{indentfirst}
\usepackage[shortlabels]{enumitem}
\usepackage{microtype}
\usepackage[colorlinks=true,linkcolor=blue]{hyperref}

\setlist{parsep=0pt,listparindent=\parindent}

\newtheorem{theorem}{Theorem}[section]
\newtheorem{lemma}{Lemma}[section]

\newtheorem{corollary}{Corollary}[theorem]
\newtheorem{definition}{Definition}[section]

\theoremstyle{remark}
\newtheorem{remark}{Remark}[section]

\newcounter{lemmaCounter}
\newcounter{theoremCounter}
\newcounter{definitionCounter}

\newenvironment{shortlemma}{\refstepcounter{lemmaCounter}
\noindent\textbf{Lemma~\thelemmaCounter.}\em}

\newenvironment{shorttheorem}{\refstepcounter{theoremCounter}
\noindent\textbf{Theorem~\thetheoremCounter.}\em}

\newenvironment{shortdefinition}{\refstepcounter{definitionCounter}
\noindent\textbf{Definition~\thedefinitionCounter.}\em}

\DeclarePairedDelimiter\abs{\lvert}{\rvert}
\DeclarePairedDelimiter\ceil{\lceil}{\rceil}
\DeclarePairedDelimiter\floor{\lfloor}{\rfloor}

\makeatletter
\let\oldabs\abs
\let\oldceil\ceil 
\let\oldfloor\floor
\def\abs{\@ifstar{\oldabs}{\oldabs*}}
\def\ceil{\@ifstar{\oldceil}{\oldceil*}}
\def\floor{\@ifstar{\oldfloor}{\oldfloor*}}
\makeatother

\AtBeginDocument{\providecommand*\colonequiv{\vcentcolon\mspace{-1.2mu}\equiv}}

\newcommand{\N}{\mathbf{N}}
\newcommand{\Z}{\mathbf{Z}}
\newcommand{\Q}{\mathbf{Q}}
\newcommand{\I}{\mathbf{I}}
\newcommand{\R}{\mathbf{R}}
\newcommand{\paren}[1]{\left( #1 \right)}
\newcommand{\set}[1]{\{#1\}}
\newcommand{\exc}[1]{\textbf{Exercise #1.}}
\newcommand{\lep}[1][L]{#1et $\epsilon > 0$ be arbitrary}
\newcommand{\seq}[1]{\left(#1\right)_{n=1}^\infty}
\newcommand{\lang}{\mathcal{L}}
\newcommand{\proves}{\vdash}
\newcommand{\nproves}{\nvdash}
\newcommand{\nmodels}{\nvDash}
\newcommand{\is}{\colonequiv}
\newcommand{\limplies}{\rightarrow}
\newcommand{\liff}{\leftrightarrow}
\newcommand{\struct}[1]{\mathfrak{#1}}
\newcommand{\utilde}[1]{\underset{\sim}{#1}}
\newcommand{\prt}[1]{\mathcal{#1}}
\newcommand{\step}[1]{\sigma_{\mathrm{#1}}}
\newcommand{\powerset}[1]{\operatorname{\mathcal{P}}\paren{#1}}
\newcommand{\hyper}[1]{#1^*}

\let\oldlog\log
\let\oldmax\max
\let\oldmin\min
\let\oldsin\sin
\let\oldcos\cos
\renewcommand{\log}[1]{\oldlog \left( #1 \right)}
\renewcommand{\max}[1]{\oldmax \left( #1 \right)}
\renewcommand{\min}[1]{\oldmin \left( #1 \right)}
\renewcommand{\sin}[1]{\oldsin \left( #1 \right)}
\renewcommand{\cos}[1]{\oldcos \left( #1 \right)}

\DeclareMathOperator{\Val}{Val}

\newcommand{\restr}[2]{\struct{#1}{{\upharpoonright}}_{\mathcal{#2}}}

\newcounter{proofCounter}
\newcommand*{\NewRow}{\stepcounter{proofCounter}\arabic{proofCounter}}%


\title{Nonstandard Analysis}
\author{Eduardo Freire}
\date{April 2021}

\begin{document}

\maketitle

\section{Ultrafilters}

\begin{definition} \label{def:filter}
    Given a set $X$ we say that a nonempty $U \in \powerset{\powerset{X}}$ is an ultrafilter if and only if
    \begin{enumerate}
        \item $\emptyset \notin U$,
        \item $A \in U \limplies A \subseteq B \limplies B \in U$,
        \item $A \in U \limplies B \in U \limplies A \cap B \in U$,
        \item $A \in U \lor X \setminus A \in U$.
    \end{enumerate}
    
    A nonempty set $F \in \powerset{\powerset{X}}$ that satisfies requirements $(2)$ and $(3)$ is called a filter. If it also satisfies requirement $(1)$ then it is a proper filter. 
\end{definition}


\begin{definition}
    A sets $A$ is said to have the finite intersection property (fip) if and only if every finite intersection of its elements is nonempty.
\end{definition}

\begin{lemma} \label{lemma:fip_insert_of_proper}
    If $F$ is a proper filter on $X$ then for any $A \subseteq X$, $F \cup \set{A}$ or $F \cup \set{X \setminus A}$ has the fip.
\end{lemma}

\begin{proof}
    One can show this by first noticing that either every element of $F$ intersects with $A$ or every element of $F$ intersects with $X \setminus A$. From that the result follows easily.
\end{proof}

\begin{lemma} \label{lemma:smallest_filter_containing}
    If $A$ is any family of subsets of $X$, then
    \begin{equation*}
        F := \set{B \subseteq X : \exists A_1, \dots, A_n \in A (B \supseteq \bigcap_{i=1}^n A_n))}
    \end{equation*} is the smallest filter containing $A$, also called the filter generated by $A$. Furthermore, if $A$ has the finite intersection property then $F$ is proper. \qed
\end{lemma}

\begin{lemma} \label{lemma:ultrafilter_iff}
    $U$ is an ultrafilter on $X$ if and only if $U$ is a maximal proper filter.
\end{lemma}

\begin{proof}
    For the forward direction assume that $U$ is an ultrafilter on $X$. We will show that every filter that properly contains $U$ is non-proper. So let $F$ be a filter properly containing $U$. Then there is some $A \in F$ such that $A \notin U$. But $U$ is an ultrafilter, hence $X \setminus A \in U \subsetneq F$, hence $X \setminus A \in F$. Then $\emptyset = A \cap (X \setminus A) \in F$, hence $F$ is not proper.
    
    For the other direction we work in the contrapositive, so assume that there is some $A \subseteq X$ such that $A \notin U$ and $X \setminus A \notin U$. By Lemma \ref{lemma:fip_insert_of_proper} one of $U \cup \set{A}$ or $U \cup \set{X \setminus A}$ has the fip. But $U$ is a proper subset of both of these sets, and by Lemma \ref{lemma:smallest_filter_containing}, the filter generated by the set with the fip is proper. Thus we obtain a proper filter which properly contains $U$, hence $U$ is not a maximal proper filter.
\end{proof}

\begin{theorem}[Ultrafilter lemma] \label{thm:ultrafilter_of_proper_filter}
    For any proper filter $F$ on $X$ there is an ultrafilter which contains it.
\end{theorem}

\begin{proof}
    We will prove this result by applying Tukey's Lemma and remark that a similar proof can be achieved using Zorn's Lemma. First, notice that the set $\set{U \subseteq \powerset{X} : \text{$U$ has the fip}}$ is of finite character, so Tukey's Lemma gives us a maximal $U \supseteq F$ which has the fip. 
    
    By Lemma \ref{lemma:smallest_filter_containing}, the filter $G$ generated by $U$ is proper, thus it also has the fip. By maximality, $U$ cannot be a proper subset of $G$, and since we already have $U \subseteq G$ we must have $U = G$, thus $U$ is a proper filter. $U$ must also be a maximal proper filter, since any proper filter has the fip. Thus, by Lemma \ref{lemma:ultrafilter_iff}, $U$ is an ultrafilter.
\end{proof}

\begin{corollary} \label{col:ultrafilter_of_fip}
    Given any $A \subseteq \powerset{X}$ with the fip there is an ultrafilter $U$ which contains it.
\end{corollary}

\begin{proof}
    This follows directly by applying Lemma \ref{lemma:smallest_filter_containing} followed by Theorem \ref{thm:ultrafilter_of_proper_filter}.
\end{proof}


\section{Ultraproducts}

\begin{definition}
    Let $I$ be an index set, $U$ be an ultrafilter on $I$, $\lang$ be a first-order signature, and for each $i \in I$ let $\struct{M}_i$ be an $\lang$-structure. We define the equivalence relation $\equiv$ on $\prod_{i \in I} M_i$ as $a \equiv b \iff \set{i \in I : a_i = b_i} \in U$.
    
    Then, we form the $\lang$-structure $\struct{M}$ whose universe is the quotient $\paren{\prod_{i \in I} M_i} / \equiv$. For each $n$-ary function symbol of $\lang$, and each $([r^1], \dots, [r^n]) \in M^n$, we define $f^\struct{M} : M^n \to M$ so that
    $$f^\struct{M}([r^1], \dots, [r^n]) = [i \mapsto f^{\struct{M}_i}(r^{1}_i, \dots, r^{n}_i)].$$ For each $n$-ary relation symbol $R$, we say that 
    
    \begin{equation*}
        ([r^1], \dots, [r^n]) \in R^\struct{M} \iff \set{i \in I : (r^1_i, \dots, r^n_i) \in R^{\struct{M}_i}} \in U.
    \end{equation*} Finally, each constant symbol $c \in \lang$ is interpreted as \begin{equation*}
        c^\struct{M} = [i \mapsto c^{\struct{M}_i}].
    \end{equation*} The structure $\struct{M}$ is called the \textbf{ultraproduct} of $\set{M_i : i \in I}$. If all of the $M_i$'s are equal, then the structure $\struct{M}$ is called the ultrapower of $M_i$.
\end{definition}

\begin{remark}
    Fix some $n \in \N$ and assume that $([r^1], \dots, [r^n]) = ([s^1], \dots, [s^n])$. Then $\set{i \in I : r^k_i = s^k_i} \in U$ for each $k \in 1, \dots, n$. Thus $A = \bigcap_{k = 1}^n \set{i \in I : r^k_i = s^k_i} \in U$. Since $A \subseteq \set{i \in I : f^\struct{M}_i(r^1_i, \dots, r^n_i) = f^\struct{M}_i(s^1_i, \dots, s^n_i)}$, the latter set is in the ultrafilter, so that $f^\struct{M}([r^1], \dots, [r^n]) = f^\struct{M}([s^1], \dots, [s^n])$. Hence, the functions of the ultraproduct are actually well-defined. Similarly, the relations are also well-defined.
\end{remark}

\begin{theorem}
    Fix an ultrafilter $U$ on $I$ and consider the ultrapower $\struct{M}$ of some structure $\struct{N}$. The substructure $\struct{N}' \subseteq \struct{M}$ with universe $N' := \set{[i \mapsto r] : r \in N}$ is isomorphic to $\struct{N}$.
\end{theorem}

\begin{proof}
    First, it is easy to check that the $N'$ is closed under the functions of $\struct{M}$, so that it is indeed a substructure. Then one can show that the function $\Phi : N \to N'$ given by
    \begin{equation*}
        \Phi (r) = [i \mapsto r]
    \end{equation*} is an isomorphism.
\end{proof}

\begin{theorem}{(\L o\'s's Theorem)}\label{thm:los}
    Consider the ultraproduct $\struct{M}$ of $\set{M_i : i \in I}$ in the signature $\lang$. Let $\phi$ be an $\lang$-formula with $n$ free variables and $([r^1], \dots, [r^n]) \in M^n$. Then 
    \begin{equation*}
        \struct{M} \models \phi[[r^1], \dots, [r^n]] \iff \set{i \in I : \struct{M}_i \models \phi [r^1_i, \dots, r^n_i]} \in U.
    \end{equation*}
\end{theorem}

\begin{proof}
 We induct on the complexity of $\phi$. For the atomic formulas, the result follows from the fact that for an $\lang$-term $t$ we have $\Val(t)_\struct{M}[[r^1], \dots, [r^n]] = [i \mapsto \Val(t)_\struct{M_i}[r^1_i, \dots, r^n_i]]$. If $\phi = \alpha \lor \beta$, then \begin{align*}
     \struct{M} \models \phi[[r^1], \dots, [r^n]] \iff && \\ \struct{M} \models \alpha [[r^1], \dots, [r^n]] \lor \struct{M} \models \beta [[r^1], \dots, [r^n]] \iff && \\ (\text{Definition of $\models$}) \\ \set{i \in I : \struct{M}_i \models \alpha [r^1_i, \dots, r^n_i]} \in U \lor \set{i \in I : \struct{M}_i \models \beta [r^1_i, \dots, r^n_i]} \in U \iff \\ (\text{Inductive hypothesis}) \\ \set{i \in I : \struct{M}_i \models \alpha [r^1_i, \dots, r^n_i]} \in U \cup \set{i \in I : \struct{M}_i \models \beta [r^1_i, \dots, r^n_i]} \in U \iff \\ (\text{$U$ is an ultrafilter}) \\ \set{i \in I : \struct{M}_i \models \alpha [r^1_i, \dots, r^n_i] \lor \struct{M}_i \models \beta [r^1_i, \dots, r^n_i]} \in U  \iff \\ \set{i \in I : \struct{M}_i \models \phi [r^1_i, \dots, r^n_i]} \in U \\ (\text{Definition of $\models$})
 \end{align*} The case $\phi = \neg \alpha$ is similar, and also uses the fact that $U$ is an ultrafilter.
 
 Next, consider the case where $\phi = \exists y \psi$. To simplify the notation we will assume that $n = 1$, since the argument remains unchanged for other $n$. By the definition of $\models$, we have $\struct{M} \models \exists y \psi [[r]] \iff \exists [s] \in M (M \models \psi[[r], [s]])$. Using the inductive hypothesis, this is equivalent to $\exists [s] \in M (\set{i \in I : \struct{M}_i \models \psi[r_i, s_i]} \in U)$. We need to show that this last statement is equivalent to $\set{i \in I : \exists m \in M_i (M_i \models \psi[r_i, m])} \in U$. The forward direction is clear, and for the converse direction we can use choice to form the element $[i \mapsto m_i] \in M$ by picking each $m_i$ so that $\struct{M_i} \models \psi[r_i, m_i]$ when there is one, and setting it to anything else when this is not the case.
\end{proof}

\begin{remark}
    Notice that the statement of the theorem does not depend on the representatives picked for each $r^k$.
\end{remark}

In the special case where $\phi$ is a sentence and $\struct{M}$ is an ultrapower of $\struct{M}'$, we have $\struct{M} \models \phi \iff \struct{M}' \models \phi$. Hence the following corollary:

\begin{corollary}
    Every structure is elementarily equivalent to its ultrapower.\qed
\end{corollary}

\begin{corollary} [Compactness Theorem]
    Let $\Sigma$ be a set of $\lang$-sentences. Then $\Sigma$ is satisfiable if and only if all of its finite subsets are satisfiable.
\end{corollary}

\begin{proof}
    The forward direction is clear, so we focus on the converse. Let $I$ be the set of all finite subsets of $\Sigma$, which are all satisfiable by assumption. For each $i \in I$ fix one $\struct{M}_i$ which models $i$. Also define for each $i \in I$ the set $A_i := \set{j \in I : j \supseteq i}$. 
    
    Notice that given any finite family $A_{i_1}, \dots, A_{i_n}$ of $A := \set{A_i : i \in I}$ the set $i_1 \cup \dots \cup i_n \in A_{i_1} \cap \dots \cap A_{i_n}$, so $A$ has the finite intersection property. Now use Corollary \ref{col:ultrafilter_of_fip} to obtain an ultrafilter $U \subseteq \powerset{I}$ which contains $A$ and let $\struct{M}$ be the ultraproduct of $\set{\struct{M}_i : i \in I}$ with respect to $U$. We will now show that $\struct{M} \models \Sigma$, which implies $\Sigma$ is satisfiable.
    
    Fix some $\phi \in \Sigma$. We need to show that $\struct{M} \models \phi$, which, by Theorem \ref{thm:los}, is equivalent to showing that $\set{i \in I : \struct{M}_i \models \phi} \in U$. Notice that $A_{\set{\phi}} \in U$ (since $A \subseteq U$) and $A_{\set{\phi}} \subseteq \set{i \in I : \struct{M}_i \models \phi}$. By the definition of ultrafilter, this implies that $\set{i \in I : \struct{M}_i \models \phi} \in U$, as we wanted to show.
\end{proof}

\section{Hyperreal Numbers}
    \begin{definition}
        Let $\lang$ be a language such that for each relation $A^n \subseteq \R^n$ there is a distinct relation symbol $R_{A}$, for each function $g : \R^n \to \R$ there is a distinct function symbol $f_{g}$ and for each $x \in \R$ there is a distinct constant symbol $c_{x}$.
        
        We will use $\R$ to denote the first order structure which interprets $\lang$ in the expected way. The hyperreal numbers $\hyper{\R}$ will be the ultrapower of $\R$ by an ultrafilter $U$ on $\N$ which contains the cofinite sets of naturals. Furthermore, whenever $A$ is a relation, constant or function of $\R$ we will use $\hyper{A}$ to denote the interpretation of $A$ in $\hyper{\R}$.
    \end{definition}
    \begin{remark}
        The ultrapower construction is justified since the family $F$ of cofinite sets on $\N$ has the fip, so Corollary \ref{col:ultrafilter_of_fip} guarantees that the desired ultrafilter $U$ which contains $F$ exists.
    \end{remark}


\end{document}
