\documentclass{report}
\usepackage[utf8]{inputenc}
\usepackage[english]{babel}
\usepackage{xparse}
\usepackage{amsmath}
\usepackage{amsthm}
\usepackage{amssymb}
\usepackage{mathtools}
\usepackage{amsfonts}
\usepackage{indentfirst}
\usepackage[shortlabels]{enumitem}
\usepackage{microtype}
\usepackage{esvect}
\usepackage[colorlinks=true,linkcolor=blue]{hyperref}

\setlist{parsep=0pt,listparindent=\parindent}

\newtheorem{theorem}{Theorem}[section]
\newtheorem{lemma}{Lemma}[section]

\newtheorem{corollary}{Corollary}[section]
\newtheorem{proposition}{Proposition}[section]
\theoremstyle{definition}
\newtheorem{definition}{Definition}[section]
\theoremstyle{remark}
\newtheorem{remark}{Remark}[section]

\newenvironment{exc}[1]{\noindent\textbf{Exercise \thesection.#1.}}{\medskip}


\DeclarePairedDelimiter\abs{\lvert}{\rvert}
\DeclarePairedDelimiter\ceil{\lceil}{\rceil}
\DeclarePairedDelimiter\floor{\lfloor}{\rfloor}

\makeatletter
\let\oldabs\abs
\let\oldceil\ceil 
\let\oldfloor\floor
\def\abs{\@ifstar{\oldabs}{\oldabs*}}
\def\ceil{\@ifstar{\oldceil}{\oldceil*}}
\def\floor{\@ifstar{\oldfloor}{\oldfloor*}}
\makeatother

\newcommand{\N}{\mathbb{N}}
\newcommand{\Z}{\mathbb{Z}}
\newcommand{\Q}{\mathbb{Q}}
\newcommand{\I}{\mathbb{I}}
\newcommand{\R}{\mathbb{R}}
\newcommand{\C}{\mathbb{C}}
\newcommand{\paren}[1]{\left( #1 \right)}
\newcommand{\set}[1]{\{#1\}}
\newcommand{\seq}[2][n \in \N]{\left( #2 \right)_{#1}}
\newcommand{\prt}[1]{\mathcal{#1}}
\newcommand{\lep}[1][L]{#1et $\epsilon > 0$ be arbitrary}
\newcommand{\ball}[3]{B_{#1}^{#3}(#2)}
\newcommand{\closure}[2][3]{%
{}\mkern#1mu\overline{\mkern-#1mu#2}}
\newcommand*\concat{\mathbin{\vcenter{\hbox{\rule{.3ex}{.3ex}}}}}
\newcommand{\powerset}[1]{\mathcal{P}\left (#1\right)}
\newcommand{\limplies}{\rightarrow}
\newcommand{\ON}{\mathrm{ON}}
\newcommand{\CARD}{\mathrm{CARD}}
\newcommand{\WF}{\mathrm{WF}}
\newcommand{\lang}{\mathcal{L}}
\newcommand{\proves}{\vdash}
\newcommand{\nproves}{\nvdash}
\newcommand{\nmodels}{\nvDash}
\newcommand{\liff}{\leftrightarrow}
\newcommand{\struct}[1]{\mathfrak{#1}}
\newcommand{\utilde}[1]{\underset{\sim}{#1}}
\newcommand{\step}[1]{\sigma_{\mathrm{#1}}}
\newcommand{\submodel}{\preccurlyeq}

\let\oldlog\log
\let\oldmax\max
\let\oldmin\min
\let\oldsin\sin
\let\oldcos\cos
\renewcommand{\log}[1]{\oldlog \left( #1 \right)}
\renewcommand{\max}[1]{\oldmax \left( #1 \right)}
\renewcommand{\min}[1]{\oldmin \left( #1 \right)}
\renewcommand{\sin}[1]{\oldsin \left( #1 \right)}
\renewcommand{\cos}[1]{\oldcos \left( #1 \right)}

\DeclareMathOperator{\type}{type}
\DeclareMathOperator{\cf}{cf}
\DeclareMathOperator{\rank}{rank}
\DeclareMathOperator{\trcl}{trcl}



\title{Set Theory}
\author{Eduardo Freire}
\date{January 2022}

\begin{document}

\maketitle
\tableofcontents

\chapter{Background Material}
\section{Ordinals}

\begin{definition}
    A set $\alpha$ is an ordinal if and only if it is transitive and well-ordered by $\in$. $ON$ is the class $\set{\alpha : \alpha \text{ is an ordinal}}$ of all ordinals.
\end{definition}

\begin{lemma}
    If $\alpha$ is an ordinal, then $\alpha \notin \alpha$.
\end{lemma}

\begin{proof}
    If $\alpha \in \alpha$ than $\in$ is not irreflexive in $\alpha$.
\end{proof}

\begin{lemma}
    $ON$ is a transitive class, that is, if $\alpha \in ON$ and $\beta \in \alpha$ then $\beta \in ON$.
\end{lemma}

\begin{proof}
    Let $\alpha$ be an ordinal and $\beta \in \alpha$. Then $\beta \subseteq \alpha$, so it is also well-ordered by $\in$. To see that $\beta$ is transitive, consider some $\gamma \in \beta$ and $x \in \gamma$. We have $\gamma \in \alpha$ so $\gamma \subseteq \alpha$, thus $x \in \alpha$. Thus $x, \gamma, \beta \in \alpha$ and $x \in \gamma \in \beta$, so $x \in \alpha$.
\end{proof}

\begin{lemma}
    If $\alpha, \beta$ are ordinals than so are $\alpha \cap \beta$ and $\alpha \cup \beta$.
\end{lemma} \qed

\begin{definition}
    Let $<$ be the relation $\in$ on the ordinals and for any set $x$ let $S(x) := x \cup \set{x}$.
\end{definition}

\begin{lemma}
    For any ordinal $\alpha$, $S(\alpha)$ is also an ordinal.
\end{lemma} \qed

\begin{lemma}
    There are no ordinals $\alpha, \beta$ with $\alpha \in \beta$ and $\beta \in \alpha$.
\end{lemma}

\begin{proof}
    Let $\alpha$ and $\beta$ be ordinals. Then $\alpha, \beta \in S(\alpha) \cup S(\beta)$ which is also an ordinal. Thus exactly one of $\alpha \in \beta$, $\beta \in \alpha$ or $\alpha = \beta$ must be true.
\end{proof}

\begin{lemma}
    If $\alpha$ is an ordinal then $S(\alpha)$ is its successor, which means $\alpha < S(\alpha)$ and $\neg \exists \beta \in ON (\alpha < \beta \land \beta < S(\alpha))$.
\end{lemma}

\begin{proof}
    Clearly $\alpha \in S(\alpha) = \alpha \cup \set{\alpha}$. Now assume that $\beta$ is an ordinal and $\alpha < \beta < S(\alpha)$. Since $\beta \in S(\alpha) = \alpha \cup \set{\alpha}$, either $\beta \in \alpha$ or $\beta = \alpha$, but both lead to a contradiction.
\end{proof}

\setcounter{section}{10}
\section{Cardinals}

\begin{theorem}
    For every set $A$ there is a cardinal $\kappa$ such that $\kappa \npreceq A$.
\end{theorem}

\begin{proof}
    Consider the set $$W := \set{(X, R) \in \powerset{A} \times \powerset{A \times A} : \text{$R$ well orders $X \times X$}}.$$ Using replacement we can make the ordinal $\beta := \sup\set{\type(X, R) + 1 : (X, R) \in W}$. Now notice that if $\alpha \preceq A$, then $\alpha = \type(X, R)$ for some $(X, R) \in W$, thus $\beta > \alpha$. Thus $\beta \npreceq A$, since otherwise we'd have $\beta < \beta$. Now $\kappa = |\beta| \approx \beta$, so $\kappa$ is a cardinal satisfying $\kappa \npreceq A$.
\end{proof}

\begin{exc}{29}
    Let $\alpha$ be such that $\alpha < \beta$. Then there is some $(X, R) \in W$ such that $\type(X, R) + 1 > \alpha$, hence $\type(X, R) \geq \alpha$, thus $\alpha \preceq \type(X, R) \preceq A$. Then if $\alpha \approx \beta$ we'd have $\beta \preceq A$, a contradiction. Thus $\beta$ is a cardinal. The argument above also shows that any cardinal less than $\beta$ injects into $A$, hence $\beta = \aleph(A)$.
\end{exc}

\begin{lemma}
    The class of cardinals is a proper class.
\end{lemma}

\begin{proof}
    Assume that $C = \set{\kappa : \text{$\kappa$ is a cardinal}}$ is a set. Then $\sup C$ exists and is a cardinal. For every $\kappa \in C$ we have $(\sup C)^+ > \sup C \geq \kappa$. But $(\sup C)^+$ is a cardinal, so $(\sup C)^+ \in C$, which implies $(\sup C)^+ > (\sup C)^+$, a contradiction.
\end{proof}

\begin{exc}{31}
    First we show that $\alpha < \beta \implies \aleph_\alpha < \aleph_\beta$. Let $\alpha$ be an arbitrary ordinal and consider the least $\beta$ such that $\aleph_\alpha \geq \aleph_\beta$ and $\alpha < \beta$. $\beta \neq 0$ since $\alpha \in \beta$. Assume that $\beta = \gamma + 1$. Then $\alpha \leq \gamma$, so assume first that $\alpha = \gamma$. In that case, $\beta = \alpha + 1$, so $\aleph_\alpha < (\aleph_\alpha)^+ = \aleph_\beta$. Now assume that $\alpha \in \gamma$. Since $\gamma \in \beta$, we must have $\aleph_\alpha < \aleph_\gamma < (\aleph_\gamma)^+ = \aleph_\beta$.
    
    For the last case, assume that $\beta$ is a limit ordinal. Then $\aleph_\beta = \sup\set{\aleph_\gamma : \gamma \in \beta}$. Since $\beta$ is a limit and $\alpha \in \beta$ we also have $\alpha + 1 \in \beta$. Then $\aleph_\beta \geq \aleph_{\alpha + 1} > \aleph_\alpha$, as we wanted to show.
    
    Now we show that for every cardinal $\kappa \geq \omega$ there is some $\xi$ such that $\kappa = \aleph_\xi$. Consider the least cardinal $\kappa \geq \omega$ such that $\neg \exists \xi (\kappa = \aleph_\xi)$. We must have $\kappa > \omega$ since $\omega = \aleph_0$. Let $X := \set{\alpha : \alpha \in \CARD\land \omega \leq \alpha < \kappa}$, where $\CARD$ is the class of cardinals. This set is nonempty as $\omega \in X$. Next, consider the function $f : X \to ON$ given by $f(\alpha) = \xi$ where $\xi$ is the unique ordinal such that $\aleph_\xi = \alpha$. Define the ordinal $\xi = \sup\set{f(\alpha) : \alpha \in X}$. If $\xi = 0$, then the only element of $X$ is $\aleph_0$, which implies that $\kappa = \aleph_1$, a contradiction. 
    
    Now assume that $\xi = \gamma + 1$. Since $\gamma < \xi$ there must be some $\alpha \in X$ with $f(\alpha) > \gamma$. Then $\xi \geq f(\alpha) > \gamma$, and $\xi$ must be $f(\alpha)$, otherwise we would have $f(\alpha)$ strictly between an ordinal and its successor. Thus $\xi = f(\alpha)$ and $\aleph_\xi = \alpha$. We know that $\kappa > \aleph_\xi$, so assume for contradiction that there is some $\beta$ with $\kappa > \beta > \aleph_\xi$.
    In that case $\beta = \aleph_\zeta$, so $f(\beta) = \zeta > \xi = f(\alpha)$ contradicting the fact that $\xi$ is a supremum. Thus $\kappa$ is the smallest cardinal greater than $\aleph_\xi$, so $\kappa = \aleph_{\xi + 1}$, a contradiction.
    
    For the last case assume that $\xi$ is a limit ordinal. Assume also that $\xi = f(\alpha)$ for some $\alpha \in X$. We claim that $\kappa = \aleph_{\xi + 1}$. Notice that since $\alpha \in X$ we have $\kappa > \alpha = \aleph_\xi$. Now assume for contradiction that there is some $\beta$ with $\kappa > \beta > \aleph_\xi$. Then we have $\beta = \aleph_\zeta$ and $\aleph_\zeta > \aleph_\xi$, so $\zeta > \xi$. But notice that $\beta \in X$ and $f(\beta) = \zeta$, so $\xi \geq f(\beta) = \zeta$, a contradiction. 
    
    Now we can assume that $\xi \neq f(\alpha)$ for every $\alpha \in X$. Since $\xi$ is a limit ordinal we have $\aleph_\xi = \sup\set{\aleph_\gamma : \gamma \in \xi}$. We then claim that $\kappa = \aleph_\xi$. First, consider some arbitrary $\gamma \in \xi$. Since $\gamma < \xi$ there must be some $\alpha$ with $f(\alpha) > \gamma$. Then $\aleph_f(\alpha) = \alpha > \aleph_\gamma$. But $\alpha \in X$, which means $\kappa > \alpha \geq \omega$, thus $\kappa > \aleph_\gamma$. Now assume that there is some $\beta$ with $\kappa > \beta$ that is also an upper bound for $\set{\aleph_\gamma : \gamma \in \xi}$. Then we must have $\beta = \aleph_\zeta$ and as $\beta \in X$ we have $\xi \geq \zeta = f(\beta)$. By assumption, we have $\xi \neq f(\beta)$, so $\xi > f(\beta) = \zeta$. As $\xi$ is a limit ordinal, we also have $\zeta + 1 < \xi$. Thus $\beta \geq \aleph_{\zeta + 1} > \aleph_\zeta = \beta$, a contradiction.
\end{exc}

\begin{lemma}\label{lem_aleph_ge}
    For every ordinal $\alpha$ we have $\alpha \le \aleph_\alpha$.
\end{lemma}

\begin{proof}
    Assume for contradiction that this is not the case and consider the least $\alpha$ satisfying $\aleph_\alpha < \alpha$. Clearly, $\alpha \neq 0$. Assume that there is some $\beta$ with $\alpha = \beta + 1$. Then $\beta < \alpha$, so $\beta \leq \aleph_\beta$. But $\beta + 1 > \aleph_{\beta + 1}$, which means $\beta \geq \aleph_{\beta+1} > \aleph_{\beta}$, a contradiction.
    
    Finally, assume that $\alpha$ is a limit ordinal. Then $\aleph_\alpha = \sup\set{\aleph_\gamma : \gamma \in \alpha}$. We will get a contradiction by showing that $\alpha \leq \aleph_\alpha$. It is enough to show that $\alpha \subseteq \aleph_\alpha$, so consider some $\gamma \in \alpha$. Since $\alpha$ is a limit ordinal, we also have $\gamma + 1 \in \alpha$, thus $\aleph_\alpha \geq \aleph_{\gamma + 1} > \aleph_{\gamma} \geq \gamma$, which implies $\gamma \in \aleph_\alpha$, as we wanted.
\end{proof}

\begin{exc}{33}
    If $\delta = \aleph_\delta$ then it is easy to see that $\gamma = \delta = \aleph_\gamma$, so assume otherwise. It can then be shown by induction using Lemma \ref{lem_aleph_ge} that $\delta_n < \delta_{n+1}$ for all $n \in \omega$. Notice first that $\gamma$ is a limit ordinal. It is not $0$, since $\delta \in \gamma$, and $\gamma$ is also not a successor, for assume otherwise. Then $\gamma = \alpha + 1$, and since $\alpha \in \gamma$ we must have some $\delta_n$ such that $\delta_n > \alpha$. Then $\alpha + 1 = \gamma \le \delta_n < \delta_{n+1}$, contradicting the fact that $\gamma$ is an upper bound for $\set{\delta_n : n \in \omega}$.
    
    Since $\gamma$ is a limit ordinal, we have $\aleph_\gamma = \sup\set{\aleph_\alpha : \alpha \in \gamma}$. We will show that $\aleph_\gamma$ is the least upper bound for $\set{\delta_n : n \in \omega}$, which means $\aleph_\gamma = \gamma$. First, notice that for each $n \in \omega$ we have $\delta_n < \delta_{n+1} \le \gamma$. Hence $\aleph_\gamma \ge \aleph_{\delta_n} = \delta_{n+1} > \delta$, so $\aleph_\gamma$ is an upper bound for $\set{\delta_n : n \in \omega}$.
    
    Now assume that $\zeta$ is an upper bound for $\set{\delta_n : n \in \omega}$. If we can show that $\zeta$ is an upper bound for $\set{\aleph_\alpha : \alpha \in \gamma}$ it will follow that $\zeta \ge \aleph_\gamma$, thus $\aleph_\gamma$ is the supremum for $\set{\delta_n : n \in \omega}$, as we wanted to show. So consider the least $\alpha \in \gamma$ such that $\zeta < \aleph_\alpha$. Since $\alpha < \gamma$ there is some $n \in \omega$ such that $\alpha \in \delta_n \in \gamma$. Then $\aleph_\alpha < \aleph_{\delta_n} = \delta_{n+1} \le \zeta$. Thus $\aleph_\alpha \le \zeta$, a contradiction.
    
    Now let $\xi$ be an ordinal such that $\xi = \aleph_\xi$. We will show that $\xi \geq \gamma$, when we take $\delta = 0$. It is enough to show that $\xi \geq \delta_n$ for every $n \in \omega$. Clearly $\xi \geq \delta_0 = 0$. Now assume that $\xi \geq \delta_n$. Then $\aleph_\xi \geq \aleph_{\delta_n}$, thus $\aleph_\xi = \xi \geq \delta_{n+1} = \aleph_{\delta_n}$, as we wanted to show.
\end{exc}

\begin{definition}
    A function $f : ON \to ON$ is normal if and only if $f(\gamma) = \sup\set{f(\alpha) : \alpha < \gamma}$ whenever $\gamma$ is a limit ordinal and $f(\alpha) < f (\alpha + 1)$ for all ordinals $\alpha$.
\end{definition}

\begin{lemma}
    If $f$ is a normal function then $\alpha < \beta \limplies f(\alpha) < f(\beta)$ and $\beta \leq f(\beta)$.
\end{lemma}

\begin{proof}
    Both cases are easily proved using transfinite induction on $\beta$.
\end{proof}

\begin{lemma} \label{lemma:normal_fixed_point}
    Let $f : ON \to ON$ be a normal function. Define $\delta_f : ON \times \omega \to ON$ recursively as $\delta_f(\alpha, 0) = \alpha$ and $\delta_f(\alpha, n+1) = f(\delta_f(\alpha, n))$. Then, for any ordinal $\alpha$, $\gamma := \sup\set{\delta_f(\alpha, n) : n \in \omega}$ is a fixed point of $f$, meaning that $f (\gamma) = \gamma$.
\end{lemma}

\begin{proof}
    We already know that $\gamma \leq f(\gamma)$, so we need to show that $f(\gamma) = \sup\set{f(\beta) : \beta \in \gamma} \leq \gamma$. So consider some $f(\beta)$ with $\beta \in \gamma$. Since $\gamma$ is a least upper bound, there is some $n \in \omega$ with $\delta_f(\alpha, n) > \beta$. Then, $f(\delta_f(\alpha, n)) = \delta_f(\alpha, n+1) > f(\beta)$. Since $\gamma \geq \delta_f(\alpha, n+1)$, we have shown that $\gamma \geq f(\beta)$ for all $\beta < \gamma$, hence $\gamma \geq f(\gamma)$, as we wanted to show.
\end{proof}

\begin{lemma} \label{lemma:next_fixed_point}
    If $f : ON \to ON$ is normal, then $\sup\set{\delta_f(0, n) : n \in \omega}$ is the smallest fixed point of $f$. Furthermore, if $\gamma$ is a fixed point of $f$, then $\sup\set{\delta_f(\gamma+1, n) : n \in \omega}$ is the smallest fixed point of $f$ that is larger than $\gamma$.
\end{lemma}

\begin{proof}
    By the Lemma~\ref{lemma:normal_fixed_point}, $\gamma_0 := \sup\set{\delta_f(0, n) : n \in \omega}$ is a fixed point of $f$. Now let $\gamma$ be any fixed point of $f$. It suffices to show that $\gamma \geq \delta_f(0, n)$ for all $n \in \omega$. Trivially, $\gamma \geq \delta_f(0, 0) = 0$. Now assume that $\gamma \geq \delta_f(0, n)$. Then $\gamma = f(\gamma) \geq f(\delta_f(0, n)) = \delta_f(0, n+1)$, as we wanted to show.
    
    Now let $\gamma$ and $\mu$ be fixed points of $f$ with $\mu > \gamma$. We know that $\zeta := \sup\set{\delta_f(\gamma+1, n) : n \in \omega}$ is also a fixed point of $f$, so we only need to show that $\mu \geq \zeta > \gamma$. Notice that since $\mu > \gamma$, then $\mu \geq \gamma+1 = \delta_f(\gamma+1, 0)$. Now assume that $\mu \geq \delta_f(\gamma+1, n)$. Then $f(\mu) = \mu \geq f(\delta_f(\gamma+1, n)) = \delta_f(\gamma+1, n+1)$. Hence $\mu \geq \zeta$. Also, $\zeta \geq \gamma + 1 > \gamma$, so $\mu \geq \zeta > \gamma$, as we wanted to show.
\end{proof}

\begin{definition}
    Given a normal function $f : ON \to ON$, define $G_f : ON \to \mathrm{ON}$ recursively as follows:
    \begin{align*}
        &G_f(0) = \sup\set{\delta_f(0, n) : n \in \omega} \\
        &G_f(\alpha + 1) = \sup\set{\delta_f(G_f(\alpha)+1, n) : n \in \omega} \\
        &G_f(\gamma) = \sup\set{G_f(\alpha) : \alpha \in \gamma} \text{ when $\gamma$ is a limit ordinal.}
    \end{align*} The subscript is dropped when $f$ is clear from the context.
\end{definition}

\begin{lemma} \label{lemma:fixed_of_G}
    Given a normal function $f$, $G_f(\alpha)$ is the $\alpha\textsuperscript{th}$ fixed point of $f$, meaning that $\type\set{\xi \in G(\alpha) : f(\xi) = \xi} = \alpha$.
\end{lemma} 

\begin{proof}
    We use transfinite induction on $\alpha$. By Lemma~\ref{lemma:next_fixed_point}, $G(0)$ is the smallest limit point of $f$, hence $\type\set{\xi \in G(\alpha) : f(\xi) = \xi} = \type(\emptyset) = 0$. If $\alpha = \beta + 1$, then fix the order isomorphism $h : \beta \to \set{\xi \in G(\beta) : f(\xi) = \xi}$. Then $h \cup \set{(\beta, G(\beta))}$ is an isomorphism between $\alpha$ and $\type\set{\xi \in G(\alpha) : f(\xi) = \xi}$ by the same lemma.
    
    It suffices to show that for limit $\gamma$, $G(\gamma)$ is a fixed point of $f$, since in that case it will, by the definition of $G$, be the smallest fixed point greater than all $\set{G(\alpha) : \alpha \in \gamma}$. So consider the smallest limit ordinal $\gamma$ such that $G(\gamma)$ is not a fixed point of $f$. We will show that $G(\gamma) \geq f(G(\gamma))$, a contradiction. Notice that $G(\gamma)$ is a limit ordinal, so $f(G(\gamma)) = \sup\set{f(\alpha) : \alpha \in G(\gamma)}$. Now fix some $\alpha \in G(\gamma)$. Since $G(\gamma)$ is a supremum, there is some $\beta \in \gamma$ such that $G(\beta) > \alpha$. Then, $G(\beta) = f(G(\beta)) > f(\alpha)$. Thus $G(\gamma) \geq G(\beta) > f(\alpha)$, meaning that $G(\gamma) \geq f(G(\gamma))$, as we wanted to show.
\end{proof}

\section{Well Founded Sets}
    In this section we work over $\mathrm{ZF^-}$, that is, $\mathrm{ZF}$ without the Axiom of Foundation.
    
    \begin{definition}[Well Founded Hierarchy]
        We define the class $\mathrm{WF}$ as $\bigcup \set{R_\alpha : \alpha \in \ON}$ where the $R_\alpha$'s are defined as follows:
        \begin{align*}
            R_0 &= \emptyset \\
            R_{\alpha+1} &= \powerset{R_\alpha} \\
            R_{\gamma} &= \bigcup_{\alpha < \gamma} R_\alpha,
        \end{align*} where $\gamma$ is a limit ordinal.
    \end{definition}
    
    Notice that if $X \in \mathrm{WF}$ then the first $\alpha$ for which $X \in \alpha$ is a successor ordinal. This allows for the next definition.
    
    \begin{definition}
        If $X \in \WF$ then we define $\rank(X)$ as $\alpha$, where $\alpha + 1$ is the least ordinal $\beta$ with $X \in R_{\beta}$.
    \end{definition}
    
    \begin{lemma} \label{lem:R_facts}
    For a given ordinal $\alpha$ and a set $X$, we have the following:
        \begin{enumerate}
            \item $R_\alpha$ is a transitive and $\WF$ is transitive;
            \item $R_\alpha = \set{x \in \WF : \rank(x) < \alpha}$;
            \item If $X \in \WF$ and $x \in X$ then $x \in \WF$ and $\rank(x) < \rank(X)$;
            \item $\trcl(X) \in \WF \iff X \in \WF$.
        \end{enumerate}
    \end{lemma}
    
    \begin{proof}
        (1) follows easily by induction. For (2) it is clear that $R_\alpha \subseteq \set{x \in \WF : \rank(x) < \alpha}$. For the other direction, let $x$ have rank $\beta < \alpha$. Then $x \in R_{\beta+1}$ and $\beta+1 \leq \alpha$. But it is easy to see that for any $\gamma, \lambda$ with $\gamma \leq \lambda$ we have $R_\gamma \subseteq R_\lambda$, so $R_{\beta+1} \subseteq R_{\alpha}$, which implies that $x \in R_{\alpha}$.
    
        (3) follows easily from (2), so we focus on (4). For the forward direction it is enough to show that if $Y \in \WF$ and $X \subseteq Y$ then $X \in \WF$, but this follows immediately from (2). For the other direction, we can use (2) to see that it suffices to show that every element of $\trcl(X)$ has rank less than $\rank(X) = \alpha$. So let $x \in \trcl(X)$. Now recall that $$\trcl(X) = \bigcup\set{{\bigcup}^n X : n \in \N},$$ so $x \in \bigcup^n X$ for some $n \in \N$. We proceed by induction, where the basis case follows from (3). For the inductive step, assume that $x \in \bigcup^{n+1} X$. Then there is some $Y$ with $x \in Y \in \bigcup^n X$. Thus $\rank(x) < \rank(Y) < \alpha$, by (3) and the inductive hypothesis respectively.
    \end{proof}
    
    \begin{lemma} \label{lem:mem_WF_of_mems_WF}
        Let $X$ be a set and assume that $\forall y \in X (y \in \WF)$. Then $X \in \WF$ and
        
        \begin{equation*}
            \rank(X) = \sup\set{\rank(y) + 1 : y \in X}.
        \end{equation*}
    \end{lemma}
    
    \begin{proof}
        Let $\alpha = \sup\set{\rank(y) + 1 : y \in X}$. Then for every $y \in X$ we have $\rank(y) < \alpha$, hence $X \subseteq R_{\alpha}$, by the second item of Lemma \ref{lem:R_facts}. Hence $X \in \powerset{R_\alpha} = R_{\alpha + 1}$, so $X \in \WF$ and $\rank(X) \leq \alpha$. Now assume that $\rank(X) < \alpha$. Then there is some $y \in X$ such that $\rank(X) < \rank(y) + 1$, which implies $\rank(X) \leq \rank(y)$. But now item 3 of Lemma \ref{lem:R_facts} guarantees that $\rank(y) < \rank(X)$, a contradiction.
    \end{proof}
    
    \begin{theorem}
        $V = \WF$ is equivalent to the Axiom of Foundation.
    \end{theorem}
    
    \begin{proof}
        A simple induction argument settles the forward direction, using the fact that the power set of any set is well founded, since it contains the empty set.
        
        For the other direction, assume the Axiom of Foundation and assume for contradiction that there is some $X \notin \WF$. Then, by Lemma \ref{lem:R_facts}, $\trcl(X) \notin \WF$. Now consider the set $W := \set{x \in \trcl(X) : x \notin \WF}$. If $W = \emptyset$ then every element of $\trcl(X)$ is in $\WF$, thus so is $\trcl(X)$, by Lemma \ref{lem:mem_WF_of_mems_WF}. Otherwise, let $m$ be the $\in$-least element of $W$. Notice that all of the elements of $m$ are in $\trcl(X)$, since $\trcl(X)$ is transitive. But since none of its elements can belong to $W$, they must all be in $\WF$. But in that case Lemma \ref{lem:mem_WF_of_mems_WF} can be applied again to let us conclude that $m \in \WF$, contradicting the fact that $m \in W$.
    \end{proof}
    
    

\setcounter{section}{12}
\section{Cardinal arithmetic}

\begin{definition}
    If $\gamma$ is a limit ordinal, then 
    \begin{equation*}
        \cf(\gamma) := \oldmin\set{\type(X) : X \subseteq \gamma \land \sup X = \gamma}.
    \end{equation*} Then, $\gamma$ is regular if and only if $\cf(\gamma) = \gamma$.
\end{definition}

\begin{theorem} \label{theorem:union_of_lt_regular}
    If $\theta$ is regular, $|F| < \theta$ and $|S| < \theta$ for all $S \in F$, then $|\bigcup F| < \theta$.
\end{theorem}

\begin{proof}
    Let $X := \set{|S| : S \in F}$. Then $X \subseteq \theta$ and $|X| < \theta$, so $\type(X) < \theta$. Since $\theta$ is regular, this implies that $\sup X < \theta$. Now let $\kappa = \max{\sup X, |F|} < \theta$. If $\kappa$ is infinite, then  $|\bigcup F| < \kappa$. If it is finite, then $|\bigcup F|$ is finite. In either case, $|\bigcup F| < \theta$. 
\end{proof}

\begin{definition}
    If $\kappa$ is a cardinal, then it its weakly inaccessible if and only if it is regular, $\kappa > \aleph_0$ and $\kappa >\alpha^+$ for all $\alpha < \kappa$. $\kappa$ is strongly inaccessible if and only if it is regular, $\kappa > \aleph_0$ and $\kappa > 2^\alpha$ for all $\alpha < \kappa$.
\end{definition}

\begin{theorem}
    If $\kappa$ is a weakly inaccessible cardinal, then it is the $\kappa\textsuperscript{th}$ element of $\set{\aleph_\alpha : \alpha = \aleph_\alpha}$. If $\kappa$ is strongly inaccessible, then it is the $\kappa\textsuperscript{th}$ element of $\set{\beth_\alpha : \alpha = \beth_\alpha}$.
\end{theorem}

\begin{proof}
    Notice that $\aleph, \beth : \mathrm{ON} \to \mathrm{ON}$ are normal functions and let $\kappa$ be weakly inaccessible. To see that $\kappa$ is an $\aleph$ fixed point, we show that $\kappa \geq \aleph_\kappa$. We have $\aleph_\kappa = \sup\set{\aleph_\alpha : \alpha \in \kappa}$, so we show by induction that $\kappa > \aleph_\alpha$ for all $\alpha \in \kappa$. Clearly, $\kappa > \aleph_0$. If $\kappa > \aleph_\alpha$ then $\kappa > (\aleph_\alpha)^+ = \aleph_{\alpha + 1}$. If $\alpha$ is a limit ordinal then $\aleph_\alpha = \sup\set{\aleph_\beta : \beta \in \alpha}$. Clearly $\kappa \geq \aleph_\alpha$, and $\kappa \neq \aleph_\alpha$, since otherwise $\set{\aleph_\beta : \beta \in \alpha}$ would be an unbounded subset of $\kappa$ with order type $\alpha$ which is less than $\kappa$, contradicting the assumption that $\kappa$ is regular.
    
    Now we have to show that $\kappa$ is the $\kappa\textsuperscript{th}$ fixed point of $\aleph$. By Lemma~\ref{lemma:fixed_of_G}, this is equivalent to showing that $G_\aleph(\kappa) = \kappa$. Since $G$ is normal, we already have $\kappa \leq G(\kappa)$, so we show that $\kappa \geq G(\kappa)$. Notice that if $\gamma$ is not a limit ordinal, then $\cf (G(\gamma)) = \omega$, hence $\kappa \neq G(\gamma)$. Then, $\kappa \geq G(0)$ as $G(0)$ is the smallest $\aleph$ fixed point, so $\kappa > G(0)$. If $\kappa > G(\alpha)$, then $\kappa \geq G(\alpha + 1)$, since $G(\alpha + 1)$ is the smallest fixed point greater than $G(\alpha)$. Since $\alpha + 1$ is a successor, $\kappa > G(\alpha + 1)$.
    
    Now let $\gamma$ be the smallest ordinal in $\kappa$ such that $\kappa < G(\gamma)$. By the argument above, $\gamma$ must be a limit ordinal. Then $G(\gamma) = \sup\set{G(\alpha) : \alpha < \gamma}$, hence $\kappa \geq G(\gamma)$, a contradiction. Hence $\kappa \geq \sup\set{G(\alpha) : \alpha \in \kappa} = G(\kappa)$, as we wanted to show.
    
    When $\kappa$ is strongly inaccessible, replace $\aleph$ by $\beth$ and $\alpha^+$ with $2^\alpha$ in the proof above.
\end{proof}

\begin{lemma}
    If $\kappa$ is a cardinal such that $\beth_\kappa = \kappa$, then $V_\kappa = H(\kappa)$.
\end{lemma}

\begin{proof}
    For any cardinal $\lambda$ it is already the case that $H(\lambda) \subseteq V_\lambda$, so we show only that $V_\kappa \subseteq H(\kappa)$. So consider some $x \in V_\kappa$. Since $\kappa$ is a limit ordinal we have some $\alpha \in \kappa$ with $x \in V_\alpha$. Then $\operatorname{trcl}(x) \in V_\alpha$, so $\operatorname{trcl}(x) \subseteq V_\alpha$. Then $\abs{\operatorname{trcl}(x)} \leq \abs{V_\alpha} \leq \beth_\alpha$. Since $\alpha < \kappa$, $\beth_\alpha < \beth_\kappa = \kappa$, hence $\abs{\operatorname{trcl}(x)} < \kappa$, so $x \in H(\kappa)$.
\end{proof}

\chapter{Model Theory}
\setcounter{section}{15}
\section{Elementary Submodels}

\begin{lemma}{(Tarski-Vaught criterion)}
Let $\struct{A}, \struct{B}$ be $\lang$-structures with $\struct{A} \subseteq \struct{B}$. The following are equivalent:

\begin{enumerate}
    \item $\struct{A} \submodel \struct{B}$.
    \item For all existential formulas $\phi(\vec{x}) = \exists y \psi(\vec{x}, y)$ with $n$ free variables and all $\vec{a} \in A^n$, if $\struct{B} \models \phi[\vec{a}]$ then $\struct{B} \models \psi[\vec{a}, b]$ for some $b \in A$.
\end{enumerate}
\end{lemma}

\begin{proof}
    (1) $\limplies$ (2): Let $\exists y \psi(\vec{x}, y)$ be an existential formula, $\vec{a} \in A^n$ and assume $\struct{B} \models \exists y \psi(\vec{x}, y)[\vec{a}]$. Then $\struct{A} \models \exists y \psi(\vec{x}, y)[\vec{a}]$ by $\struct{A} \submodel \struct{B}$, so there is some $b \in A$ with $\struct{A} \models psi(\vec{x}, y)[\vec{a}, b]$. Then $\struct{A} \models \psi(\vec{x}, y)[\vec{a}, b]$, by $\struct{A} \submodel \struct{B}$.
    
    (2) $\limplies$ (1): Inducting on the length of $\phi$, we only need to check the cases where $\phi$ begins with a quantifier. If $\phi = \exists y \psi(\vec{x}, y)$, then 
    \begin{equation*}
        \struct{A} \models \phi[\vec{a}] \iff \exists b \in A (\struct{A} \models \psi(\vec{x}, y)[\vec{a}, b]) \iff \struct{B} \models \phi[\vec{a}].
    \end{equation*}
\end{proof}

\begin{theorem} ({Downward Löwenheim-Skolem-Tarski Theorem})
    Let $\struct{B}$ be an infinite $\lang$-structure, and let $\kappa$ be a cardinal with $\max{\aleph_0, |\lang|} \leq \kappa \leq |B|$. Fix $S \subseteq B$ with $|S| \leq \kappa$. Then there is an $\lang$-structure $\struct{A} \submodel \struct{B}$ such that $|A| = \kappa$ and $S \subseteq A$.
\end{theorem}

\end{document}
