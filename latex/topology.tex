\documentclass{report}
\usepackage[utf8]{inputenc}
\usepackage[english]{babel}
\usepackage{xparse}
\usepackage{amsmath}
\usepackage{amsthm}
\usepackage{amssymb}
\usepackage{mathtools}
\usepackage{amsfonts}
\usepackage{indentfirst}
\usepackage[shortlabels]{enumitem}
\usepackage{microtype}
\usepackage[colorlinks=true,linkcolor=blue]{hyperref}

\setlist{parsep=0pt,listparindent=\parindent}

\newtheorem{theorem}{Theorem}[section]
\newtheorem{lemma}{Lemma}[section]

\newtheorem{corollary}{Corollary}[section]
\newtheorem{proposition}{Proposition}[section]
\theoremstyle{definition}
\newtheorem{definition}{Definition}[section]
\theoremstyle{remark}
\newtheorem{remark}{Remark}[section]

\newenvironment{exc}[1]{\noindent\textbf{Exercise \thesection.#1.}}{\medskip}


\DeclarePairedDelimiter\abs{\lvert}{\rvert}
\DeclarePairedDelimiter\ceil{\lceil}{\rceil}
\DeclarePairedDelimiter\floor{\lfloor}{\rfloor}

\makeatletter
\let\oldabs\abs
\let\oldceil\ceil 
\let\oldfloor\floor
\def\abs{\@ifstar{\oldabs}{\oldabs*}}
\def\ceil{\@ifstar{\oldceil}{\oldceil*}}
\def\floor{\@ifstar{\oldfloor}{\oldfloor*}}
\makeatother

\newcommand{\N}{\mathbb{N}}
\newcommand{\Z}{\mathbb{Z}}
\newcommand{\Q}{\mathbb{Q}}
\newcommand{\I}{\mathbb{I}}
\newcommand{\R}{\mathbb{R}}
\newcommand{\C}{\mathbb{C}}
\newcommand{\paren}[1]{\left( #1 \right)}
\newcommand{\set}[1]{\{#1\}}
\newcommand{\seq}[2][n \in \N]{\left( #2 \right)_{#1}}
\newcommand{\prt}[1]{\mathcal{#1}}
\newcommand{\lep}[1][L]{#1et $\epsilon > 0$ be arbitrary}
\newcommand{\ball}[3]{B_{#1}^{#3}(#2)}
\newcommand{\closure}[2][3]{%
{}\mkern#1mu\overline{\mkern-#1mu#2}}
\newcommand*\concat{\mathbin{\vcenter{\hbox{\rule{.3ex}{.3ex}}}}}

\let\oldlog\log
\let\oldmax\max
\let\oldmin\min
\let\oldsin\sin
\let\oldcos\cos
\renewcommand{\log}[1]{\oldlog \left( #1 \right)}
\renewcommand{\max}[1]{\oldmax \left( #1 \right)}
\renewcommand{\min}[1]{\oldmin \left( #1 \right)}
\renewcommand{\sin}[1]{\oldsin \left( #1 \right)}
\renewcommand{\cos}[1]{\oldcos \left( #1 \right)}


\title{Topology}
\author{Eduardo Freire}
\date{August 2021}

\begin{document}

\maketitle
\tableofcontents

\part{General Topology}
\chapter{Metric Spaces}
\section{Basics}

\begin{definition}
   Let $X$ be a set and $d: X \times X \to \R$ be a function. We say that $(X, d)$ is a metric space if and only if for all $x,y,z \in X$,
   \begin{enumerate}
       \item $d(x, y) = 0 \iff x = y$,
       \item $d(x, y) = d(y, x)$,
       \item $d(x, y) \leq d(x, z) + d(z, y)$.
   \end{enumerate}
\end{definition}

\begin{remark}
    Notice that on any metric space $(X, d)$ we have $d(x, y) \geq 0$ for all $x, y \in X$, since
    $0 = d(x, x) \leq d(x, y) + d(y, x) = 2 d(x, y)$.
\end{remark}

Throughout this section $(X, d)$ will be an arbitrary metric space.

\begin{definition}
   We will call a function $f : \N \to X$ a sequence in $X$. In that case, we will sometimes write $f_n$ instead of $f(n)$. When $X$ is clear from the context, we might also write $f = \seq{a_n}$ to mean that $f$ is a sequence in $X$ where $f(n) = a_n$ for each $n \in \N$.
\end{definition}

\begin{definition}
   A sequence $x: \N \to X$ is Cauchy if and only if for all $\epsilon > 0$ there is a natural number $N$ such that for all naturals $n,m \geq N$ we have $d(x_n, x_m) < \epsilon$. We also define the set $\prt{C}(X) := \set{x: \N \to X \mid \text{$x$ is Cauchy}}$ of Cauchy sequences of $X$.
\end{definition}


\begin{definition}
    A sequence $x: \N \to X$ converges if and only if there is some $L \in X$ such that $\lim_{n \to \infty} d(x_n, L) = 0$. In that case, we say that $x$ converges to $L$ or that the limit of $x$ is $L$. 
    
    Furthermore, we say that a metric space is complete if and only if all of its Cauchy sequences converge.
\end{definition}

\begin{lemma}
    Every convergent sequence is Cauchy.
\end{lemma}

\begin{proof}
    Let $x: \N \to X$ be a sequence that converges to $L \in X$. Now let $\epsilon > 0$ be arbitrary and choose $N \in \N$ such that $d(x_n, L) < \epsilon/2$ for all $n \geq N$. Then,
    
    \begin{equation*}
        d(x_n, x_m) \leq d(x_n, L) + d(L, x_m) < \epsilon/2 + \epsilon/2 = \epsilon
    \end{equation*} for all $n,m \geq N$. Thus $x$ is Cauchy, as we wanted to show.
\end{proof}

\begin{lemma}
    The limit of a Cauchy sequence is unique.
\end{lemma}

\begin{proof}
    Assume for contradiction that there is a Cauchy sequence $x: \N \to X$ and $L, L'$ with $L \neq L'$ such that $x$ converges to both $L$ and $L'$. Since $d(L, L') > 0$, we must have some $N_1 \in \N$ such that $d(x_n, L) < d(L, L')/2$ for all $n \geq N_1$ and some $N_2 \in \N$ such that $d(x_n, L') < d(L, L')/2$ for all $n \geq N_2$. So let $N := \max{N_1, N_2}$ and fix some $n \geq N$.
    
    We have that $d(x_n, L) < d(L, L')/2$ and $d(x_n, L') < d(L, L')/2$. Summing the inequalities we get that $d(L, x_n) + d(x_n, L') < d(L, L')$. But, by the triangle inequality, $d(L, L') \leq d(L, x_n) + d(x_n, L')$, a contradiction. 
\end{proof}

\begin{remark}
    Not every metric space is complete. Consider for example $Q = (\Q, d)$, where $d: \Q \times \Q \to \R$ is given by $d(p, q) = \abs{p - q}$ for all $p, q \in \Q$. Clearly, $Q$ is a metric space, but the Cauchy sequence
    
    \begin{equation*}
        (3, 3.1, 3.14, 3.141, 3.1415, 3.14159, \dots)
    \end{equation*} does not converge, since $\pi$ is irrational.
\end{remark}

\begin{definition}
   We will say that two sequences $x,y: \N \to X$ are equivalent if and only if $\lim_{n \to \infty} d(x_n, y_n) = 0$. This defines an equivalence relation $\sim$ on $\prt{C}(X)$, namely $x \sim y \iff \text{$x$ is equivalent to $y$}$.
\end{definition}

\begin{remark}
     It is obvious that $\sim$ is reflexive and symmetric, so we check only that it is transitive. Assume that $x, y, z \in \prt{C}(X)$ and $x\sim y$ and $y \sim z$. Let $\epsilon > 0$ be arbitrary. Choose $N_1 \in \N$ such that $d(x_n, y_n) < \epsilon/2$ for all $n \geq N_1$ and $N_2 \in \N$ such that $d(y_n, z_n) < \epsilon/2$ for all $n \geq N_2$ and set $N := \max{N_1, N_2}$. For any $n \geq N$ we have $d(x_n, z_n) \leq d(x_n, y_n) + d(y_n, z_n) < \epsilon/2 + \epsilon/2 = \epsilon$, so $x \sim z$ as we wanted to show.
\end{remark}

\begin{lemma} \label{lem_equivIsCauchy}
    If $x \in \prt{C}(X)$ is equivalent to $y: \N \to X$, then $y$ is also Cauchy.
\end{lemma}

\begin{proof}
    Let $\epsilon > 0$ be arbitrary. Choose $N$ large enough so that $d(x_n, y_n) < \epsilon/3$ and $d(x_n, x_m) < \epsilon/3$ for all $n,m \geq N$. Now let $n,m \geq N$ be arbitrary. Then, we have
    
    \begin{align*}
        &d(y_n, y_m) \leq d(y_n, x_n) + d(x_n, y_m) \\
        &\leq d(y_n, x_n) + d(x_n, x_m) + d(x_m, y_m) \\
        &< \epsilon/3 + \epsilon/3 + \epsilon/3 = \epsilon,
    \end{align*} so $y$ is Cauchy, as we wanted to show.
\end{proof}

\begin{lemma} \label{lem_equivConvergeToSameLimit}
    If a sequence $x$ converges and $x \sim y$, then $y$ converges to the same limit as $x$.
\end{lemma}

\begin{proof}
    Let $x, y: \N \to X$ and assume that $x \sim y$ and $\lim x = L$. Notice that for all $n \in \N$ we have $0 \leq d(y_n, L) \leq d(y_n, x_n) + d(x_n, L)$. By the Squeeze Theorem we can conclude that $y$ converges to $L$.
\end{proof}

\section{Completing a Metric Space}

\begin{definition}
   Let $\Tilde{X}$ denote the set of all equivalence classes of $\prt{C}(X)$ under $\sim$, namely 
       $\Tilde{X} := \set{[x] \mid  x \in \prt{C}(X)}$, where $[x] = \set{y \in \prt{C}(x) \mid x \sim y}$. We also define the function $\Tilde{d}: \Tilde{X} \times \Tilde{X} \to \R$ as $\Tilde{d}([x], [y]) = \lim_{n \to \infty} d(x_n, y_n)$ for all $x, y \in \prt{C}(X)$.
\end{definition}

\begin{lemma} \label{lem_metricDFunc}
    The function $\Tilde{d}$ is well-defined
\end{lemma}

\begin{proof}
    First we show that if the sequences $\seq{x_n}, \seq{y_n}$ are Cauchy, then $\lim_{n \to \infty} d(x_n, y_n)$ exists.
    Let $\epsilon > 0$ be arbitrary. Since $\seq{x_n}$ is Cauchy, we can choose $N_1 \in \N$ such that $d(x_n, x_m) < \epsilon/2$ for all $n, m \geq N_1$. Similarly, we can choose $N_2 \in \N$ such that $\seq{y_n}$ satisfies the analogous condition.
    
    Now set $N := \max{N_1, N_2}$ and fix arbitrary $n,m \geq N$. Notice that $d(x_n,y_n)-d(x_m, y_n) \leq d(x_n, x_m)$ and $d(x_m, y_n)-d(x_n,y_n) \leq d(x_n, x_m)$, so $\abs{d(x_m, y_n)-d(x_n,y_n)} \leq d(x_n, y_m) < \epsilon/2$. Similarly, $\abs{d(x_m, y_n)-d(x_m, y_m)} \leq d(y_n, y_m) < \epsilon/2$. Thus, we have 
    \begin{align*}
        &\abs{d(x_n, y_n) - d(x_m, y_m)} \leq \abs{d(x_n, y_n) - d(x_m, y_n)} + \abs{d(x_m, y_n) - d(x_m, y_m)} \\
        &< \epsilon/2 + \epsilon/2 = \epsilon,
    \end{align*} so $(d(x_n, y_n))$ is a Cauchy sequence of reals, and therefore converges.
    
    Next, assume that $a,b,x,y \in \prt{C}(X)$ and $a \sim x$ and $b \sim y$. In order to show that $\Tilde{d}([a], [b]) = \Tilde{d}([x], [y])$ we will show that the Cauchy sequences of reals $(d(x_n, y_n))$ and $(d(a_n, b_n))$ are equivalent. To do that, let $\epsilon > 0$ be arbitrary.
    
    Using the fact that $x$ is equivalent to $a$ and $y$ is equivalent $b$, pick $N \in \N$ such that $d(x_n, a_n) < \epsilon/2$ and $d(y_n, b_n) < \epsilon/2$ for all $n \geq N$. Now fix some $n \geq N$ and, similarly to before, we have $\abs{d(x_n, y_n)-d(a_n, y_n)} \leq d(x_n, a_n) < \epsilon/2$ and $\abs{d(a_n, y_n) - d(a_n, b_n)} \leq d(y_n, b_n) < \epsilon/2$, thus
    \begin{align*}
        &\abs{d(x_n, y_n) - d(a_n, b_n)} \leq 
        \abs{d(x_n, y_n) - d(a_n, y_n)} + 
        \abs{d(a_n, y_n) - d(a_n, b_n)}
        \\ &< \epsilon/2 + \epsilon/2 = \epsilon.
    \end{align*}
\end{proof}

\begin{remark}
     $(\Tilde{X}, \Tilde{d})$ is a metric space. The three conditions that $\Tilde{d}$ must hold follow easily from Lemma \ref{lem_metricDFunc}.
\end{remark}

\begin{definition}
   An element $[x] \in \Tilde{X}$ is called rational if and only if $x \sim y$ where $y \in \prt{C}(X)$ is a constant Cauchy sequence. We also say that a sequence in $\Tilde{X}$ is rational if and only if all of its elements are rational.
\end{definition}

\begin{lemma} \label{lem_rationalSequencesConverge}
    Every rational sequence in $\prt{C}(\Tilde{X})$ converges.
\end{lemma}

\begin{proof}
    Consider a rational sequence $\seq{[x_n]} \in \prt{C}(\Tilde{X})$. Since each element is rational, we can fix for each $n \in \N$ some constant sequence $y_n \in \prt{C}(X)$ such that $y_n \sim x_n$. We claim that $\seq{[x_n]}$ converges to $[\seq{y_n(1)}]$. Notice that since $x_n \sim y_n$, we have $[x_n] = [y_n]$ for each $n \in \N$, so it suffices to show that $\seq{[y_n]}$ converges to $[\seq{y_n(1)}]$.
    
    So we have to show that 
    \begin{equation*}
        \lim_{n \to \infty} \Tilde{d}([y_n], [\seq{y_n(1)}]) = \lim_{n \to \infty} \lim_{m \to \infty} 
        d(y_n(1), y_m(1)) = 0,
    \end{equation*} so let $\epsilon > 0$ be arbitrary. Use the fact that $\seq{y_n}$ is Cauchy to choose an $N \in \N$ such that $\Tilde{d}([y_n], [y_m]) < \epsilon/2$ for all $n,m \geq N$. Since each $y_n$ is constant, we have $\Tilde{d}([y_n], [y_m]) = d(y_n(1), y_m(1))$. Fix some $n \geq N$ and notice that $d(y_n(1), y_m(1)) < \epsilon/2$ for all $m \geq N$. Thus $\lim_{m \to \infty} d(y_n(1), y_m(1)) \leq \epsilon/2 < \epsilon$.
    
\end{proof}

\begin{lemma} \label{lem_cauchySequenceEquivToRationalSequence}
    In $(\Tilde{X}, \Tilde{d})$, every sequence is equivalent to a rational sequence.
\end{lemma}

\begin{proof}
    Let $f \in \prt{C}(\Tilde{X})$ be an arbitrary sequence.
    For each $n \in \N$, we have $f(n) = [x_n]$ where $x_n \in \prt{C}(X)$. Then, there is some $K_n \in \N$ such that $d(x_n(K_n), x_n(m)) < 1/n$ for all $m \geq K_n$, since $x_n$ is Cauchy. Then, let $g: \N \to \Tilde{X}$ be the sequence given by 
    \begin{align*}
        g(n) &= [(x_n(K_n), x_n(K_n), x_n(K_n), \dots)] \\
             &= [\seq[m \in \N]{x_n(K_n)}].
    \end{align*} It is clear that $g$ is a rational sequence by construction. To see that $g$ is equivalent to $f$ we will first show that for each $n \in \N$ we have 
    
    \begin{equation*}
        \lim_{m \to \infty} d(x_n(m), x_n(K_n)) \leq 1/n.
    \end{equation*} To do this, let $n \in \N$ be arbitrary and notice that by the construction of $K_n$, we have that $0 \leq d(x_n(m), x_n(K_n)) < 1/n \leq 1/n$ for all $m \geq K_n$. Applying the squeeze theorem gets us the desired result. Notice that since $\Tilde{d}([x_n], g(n)) = \lim_{m \to \infty} d(x_n(m), x_n(K_n))$, we have shown that $\Tilde{d}([x_n], g(n)) \leq 1/n$ for each $n \in \N$.
    
    The main result then follows easily. We have that $f$ is equivalent to $g$ if and only if $\lim_{n \to \infty} \Tilde{d}([x_n], g(n)) = 0$, but $0 \leq \Tilde{d}([x_n], g(n))  \leq 1/n$ for each $n \in \N$, so applying the squeeze theorem one more time finishes the proof.
\end{proof}

\begin{theorem}
    The metric space $(\Tilde{X}, \Tilde{d})$ is complete.
\end{theorem}

\begin{proof}
    Consider an arbitrary Cauchy sequence $f \in \prt{C}(\Tilde{X})$. By Lemma \ref{lem_cauchySequenceEquivToRationalSequence}, $f$ is equivalent to a rational sequence $g \in \prt{C}(\Tilde{X})$. Notice that $g$ must also be Cauchy, by Lemma \ref{lem_equivIsCauchy}. But then Lemma \ref{lem_rationalSequencesConverge} guarantees that $g$ converges, so $f$ must converge by Lemma \ref{lem_equivConvergeToSameLimit}.
\end{proof}

\chapter{Topological Spaces and Continuous Functions}

\setcounter{section}{11}
\section{Topological Spaces}


\begin{definition}\label{def_topology}
    A topology $\prt{T}$ on a set $X$ is a collection of subsets of $X$ satisfying the following conditions: 
    
    \begin{enumerate}
        \item $\emptyset, X \in \prt{T}$,
        \item If $U_\lambda \in \prt{T}$ for every $\lambda \in \Lambda$, then $\paren{\bigcup_{\lambda \in \Lambda} U_\lambda} \in \prt{T}$ and
        \item If $A, B \in \prt{T}$, then $A \cap B \in \prt{T}$.
    \end{enumerate}
    A subset $U$ of $X$ is called open if and only if $U \in \prt{T}$.
\end{definition}

\begin{definition}
    Let $\prt{T}, \prt{T'}$ be topologies on $X$. We say that $\prt{T'}$ is finer than $\prt{T}$ if and only if $\prt{T} \subset \prt{T'}$. Similarly, $\prt{T'}$ is coarser than $\prt{T}$ if and only if $\prt{T'} \subset \prt{T}$. 
\end{definition}

\section{Basis for a Topology}

\begin{definition}\label{def_basis}
    A collection $\prt{B}$ of subsets of $X$ is called a basis for a topology on $X$ if and only if it satisfies the following conditions:
    \begin{enumerate}
        \item $X = \bigcup_{B \in \prt{B}} B$ and
        \item If $B_1, B_2 \in \prt{B}$ and $x \in B_1 \cap B_2$, then there is some $B_3 \in \prt{B}$ such that $x \in B_3$ and $B_3 \subset B_1 \cap B_2$.
    \end{enumerate}
    Given a basis $\prt{B}$ on a set $X$, let $\prt{T}$ be the set such that $U \in \prt{T}$ if and only if for every $x \in U$ there is some $B_x \in \prt{B}$ such that $x \in B_x$ and $B_x \subset U$. We call $\prt{T}$ the set generated by $\prt{B}$.
\end{definition}

\begin{proposition}
    Let $X$ be a set and $\prt{B}$ be a a basis for $X$. The set $\prt{T}$ generated by $\prt{B}$ is a topology on $X$.
\end{proposition}

\begin{proof}
    It is easy to see that clauses $1$ and $2$ in Definition \ref{def_topology} hold using the first clause in the definition of a basis. For the last clause, assume that $A,B \in \prt{T}$ and let $x \in A \cap B$ be arbitrary. Since $A$ and $B$ are in $\prt{T}$, there are $B_1, B_2 \in \prt{B}$ such that $x \in B_1 \subset A$ and $x \in B_2 \subset B$. It follows that $x \in B_1 \cap B_2 \subset A \cap B$. By clause 2 in the definition of a basis, there is some $B_3 \in \prt{B}$ such that $x \in B_3 \subset B_1 \cap B_2 \subset A \cap B$, thus $A \cap B \in \prt{T}$.
\end{proof}

\begin{lemma} \label{lemma_basis_makes_topology}
    Let $\prt{B}$ be the basis for a topology $\prt{T}$ on $X$ (so $\prt{T}$ is the topology generated by $\prt{B}$). Then $\prt{T}$ is the collection of all unions of elements of $\prt{B}$.
\end{lemma}

\begin{proof}
    Let $U \in \prt{T}$ be arbitrary. We wish to show that there is some collection of elements in $\prt{B}$ such that their union is $U$. By Definition \ref{def_basis}, for each $x \in U$ we can choose some $B_x \in \prt{B}$ such that $x \in B_x \subset U$. It is straightforward to see that $\bigcup_{x \in U} B_x = U$. Also, since the elements of $\prt{B}$ are subsets of $X$, it is evident that their union is a subset of $X$, and the result follows.
\end{proof}

\begin{lemma} \label{lemma_finer_by_basis}
    Let $\prt{B}, \prt{B'}$ be basis for the topologies $\prt{T}, \prt{T'}$ respectively on a set $X$. Then the following are equivalent:
    
    \begin{enumerate}
        \item $\prt{T'}$ is finer than $\prt{T}$,
        \item For every $B \in \prt{B}$ and every $x \in B$, there is some $B' \in \prt{B'}$ such that $x \in B' \subset B$.
    \end{enumerate}
\end{lemma}

\begin{proof}
    For the forward direction assume (1), i.e that $\prt{T} \subset \prt{T'}$. Let $B \in \prt{B}$ and $x \in B$ be arbitrary. It is easy to see that every element of a basis for a topology is an open set in that topology, specifically $B \in \prt{T}$. Thus $B \in \prt{T'}$, and definition \ref{def_basis} guarantees that there is some $B'_x \in \prt{B'}$ such that $x \in B' \subset B$, as we wanted to show.
    
    Now assume clause number (2) and let $U \in \prt{T}$ be arbitrary. We need to show that $U \in \prt{T'}$, so let $x \in U$ be arbitrary. We know, since $\prt{T}$ is generated by $\prt{B}$, that there is some $B \in \prt{B}$ such that $x \in B \subset U$. By (2), there is also some $B' \in \prt{B'}$ such that $x \in B' \subset B \subset U$. Since $\prt{T'}$ is generated by $\prt{B'}$, this means $U \in \prt{T'}$, as we wanted to show.
\end{proof}


\begin{lemma} \label{lemma_collection_is_basis}
    Let $X$ be a set and $\prt{T}$ be a topology on $X$. If $\prt{C}$ is a collection of open sets of $X$ such that for every $U \in \prt{T}$ and every $x \in U$ there is some $C \in \prt{C}$ such that $x \in C \subset U$, then $\prt{C}$ is a basis on $X$. Furthermore, the topology generated by $\prt{C}$ is $\prt{T}$.
\end{lemma}

\begin{proof}
    Assume the hypothesis in the lemma. To show that $\prt{C}$ meets clause (1) of definition \ref{def_basis}, we need to show that for any given $x \in X$ there is some $C \in \prt{C}$ such that $x \in C$, so let $x$ be arbitrary. We now that $X$ is open, so the hypothesis of the lemma guarantees that there is some $c \in \prt{C}$ with $x \in C$.
    
    Next, assume that $C_1, C_2 \in \prt{C}$ and $x \in C_1 \cap C_2$. Since $C_1, C_2$ are open, their intersection must also be open. By the lemma hypothesis, there is some $C_3 \in \prt{C}$ such that $x \in C_3 \subset C_1 \cap C_2$, so clause (2) of definition \ref{def_basis} is met and $\prt{C}$ is a basis for $\prt{T}$.
    
    Now let the collection of subsets $\prt{T'}$ be such that $U' \in \prt{T'}$ if and only if for every $x \in U'$ there is some $C_x \in \prt{C}$ such that $x \in C_x \subset U'$. We need to show that $\prt{T} = \prt{T'}$. Assume first that $U \in \prt{T}$. The lemma hypothesis guarantees that for any $x \in U$ there is some $C \in \prt{C}$ such that $x \in C \subset U$, thus $U \in \prt{T'}$. By Lemma \ref{lemma_basis_makes_topology}, $\prt{T'}$ is the collection of all unions of elements of $\prt{C}$. So given some $U' \in \prt{T'}$, $\prt{T'}$ is some arbitrary union of elements in $\prt{C}$, but every $C \in \prt{C}$ is open, so their union is also open. This means that $U' \in \prt{T'}$, thus $\prt{T} = \prt{T'}$.
\end{proof}

\begin{definition}\label{def_subbasis}
    A subbasis $\prt{S}$ for a topology on $X$ is a collection of subsets of $X$ such that for every $x \in X$ there is some $S \in \prt{S}$ such that $x \in S$. The topology generated by $\prt{S}$ is collection of all the arbitrary unions of finite intersections of elements of $\prt{S}$.
\end{definition}

\begin{remark}
     It might not be clear at first that the set generated by $\prt{S}$ is a topology on $X$. To see that it is, notice that the collection of all finite intersections of elements of $\prt{S}$ is a basis $\prt{B}$. Then, the collection of all arbitrary unions of elements of $\prt{B}$ is the topology generated by $\prt{B}$, according to Lemma \ref{lemma_basis_makes_topology}.
\end{remark}

\section{The Order Topology}

\begin{definition}\label{def_order_basis}
    Let $X$ be a set with more than one element and $<$ be a strict linear order on $X$. We define the set $\prt{B}$ by 
    \begin{align*}
        \prt{B} := &\set{(x, y) : x < y} \cup \\ &\set{[x_0, y) : x_0 < y\text{, if $X$ has a least element $x_0$.}} \cup \\ 
        &\set{(x, y_0] : x < y_0\text{, if $X$ has a largest element $y_0$.}}
    \end{align*}
    We call $\prt{B}$ the order basis on $X$ with order $<$, and the topology it generates is called the order topology.
\end{definition}

\begin{proposition}
    Given any $X$ with more than one element and some strict linear order $<$ on $X$, the order basis $\prt{B}$ is a basis for a topology on $X$.
\end{proposition}

\begin{proof}
    Let $x \in X$ be arbitrary. We know that there is some $y \in X$ other than $x$. If $x < y$ and $x$ is the least element of $x$, then $x \in [x, y) \in \prt{B}$, otherwise there is some $z \in X$ such that $z < x < y$, thus $x \in (z, y) \in \prt{B}$. Similarly, we can show that when $y < x$ there is some $B$ such that $x \in B \in \prt{B}$.
    
    Now let $B_1, B_2 \in \prt{B}$ and $x \in B_1 \cap B_2$ be arbitrary. It is straightforward but tedious to check that there is some $B_3 \in \prt{B}$ such that $x \in B_3 \subset B_1 \cap B_2$.
\end{proof}

\begin{definition}
    The standard topology on $\R$ is the one generated by the basis $\prt{B} = \set{(a, b) : a < b}$.

    The lower limit topology $\R_{\prt{L}}$ on $\R$ is the topology generated by the basis $\prt{B'} = \set{[a, b) : a < b}$.
\end{definition}

\begin{lemma}
    The lower limit topology on the reals is strictly finer then the standard topology.
\end{lemma}

\begin{proof}
    To show that $\R_{\prt{L}}$ is finer than the standard topology, it suffices to show that given any $B$ in the standard basis $\prt{B}$ and any $x \in \R$, there is some $B' \in \prt{B'}$ such that $x \in B' \subset B$, by Lemma \ref{lemma_finer_by_basis}. So let $B \in \prt{B}$ and $x \in \R$ be arbitrary. We know that $B = (a, b)$ with $a < b$ and $a < x < b$. Then $x \in [x, b) \in \prt{B'}$ and $[x, b) \subset B$, as we wanted to show.
    
    Also, the interval $[0, 1)$ is open in the lower limit topology, but not in the standard. To see that, assume for contradiction that $[0, 1)$ is open in the standard topology. Then, there must be some $(a, b) \in \prt{B}$ such that $0 \in (a, b) \subset [0, 1)$. Since $0 \in (a, b)$, $a < 0 < b$. Then $a < a/2 < 0 < b$, so $a/2 \in (a, b)$, therefore $a/2 \in [0, b)$. Thus $a/2 \geq 0$, a contradiction. Thus, $\R_{\prt{L}}$ is strictly finer than the standard topology.
\end{proof}

\section{The Product Topology on \texorpdfstring{$X \times Y$}{X x Y}}

\begin{definition}
    Let $X$ and $Y$ be topological spaces. The product topology $X \times Y$ is defined as the topology generated by the basis $\prt{B} = \set{U \times V : \text{$U$ is open in $X$ and $V$ is open in $Y$}}.$
\end{definition}

\begin{lemma}
    Let $\prt{B}_x, \prt{B}_y$ be basis for $X$ and $Y$ respectively. It follows that $\prt{B}_x \times \prt{B}_y$ generates the product topology $X \times Y$.
\end{lemma}

\begin{proof}
    We apply Lemma \ref{lemma_collection_is_basis} to the collection $\prt{B}_x \times \prt{B}_y$ of open sets. Let $W$ be open in $X \times Y$ and $a \times b \in W$ be arbitrary. By the definition of the order topology, there is some $B \in \prt{B}$ such that $a \times b \in U \times V \subset W$, where $\prt{B}$ is the basis for $X \times Y$. Since $\prt{B}_x$ is a basis for $X$, there is some $B_x \in \prt{B}_x$ such that $a \in B_x \subset U$. Similarly, there is some $B_y \in \prt{B}_y$ such that $b \in B_y \subset V$. Then $a \times b \in B_x \times B_y \subset U \times V \subset W$, thus the conditions of the lemma just mentioned are met and $\prt{B}_x \times \prt{B}_y$ is a basis and generates the product topology.
\end{proof}

\section{The Subspace Topology}

\begin{definition}\label{def_subspace_topology}
    Let $(X, \prt{T}_x)$ be a topological space. For any $Y \subset X$, we define the subspace topology as $\prt{T}_y = \set{Y \cap U : U \in \prt{T}_x}$.
\end{definition}

\begin{lemma}
    The set constructed in definition \ref{def_subspace_topology} is a topology on $X$.
\end{lemma}

\begin{proof}
    By definition, $X \in \prt{T}_x$, so $Y \cap X = Y \in \prt{T}_y$, and similarly for the empty set. Now let $\set{Y \cap U_\lambda : \lambda \in \Lambda}$ be a collection of open sets in $\prt{T}_y$. Then 
    \begin{equation*}
        \bigcup_{\lambda \in \Lambda} Y \cap U_{\lambda} = \paren{Y \cap  \bigcup_{\lambda \in \Lambda} U_{\lambda}} \in \prt{T}_y, 
    \end{equation*} since arbitrary union of sets in $\prt{T}_x$ are open. A similar argument shows that finite intersections of sets in $\prt{T}_y$ are also in $\prt{T}_y$.
\end{proof}

\begin{lemma}
    Let $X$ be a topological space and $Y$ be the subspace topology on $X$ generated by $Y \subset X$. If $\prt{B}_x$ is a basis for $X$ then $\prt{B}_y = \set{Y \cap B_x : B_x \in \prt{B}_x}$ is a basis for $Y$.
\end{lemma}

\begin{proof}
    Let $U_y \in \prt{T}_y$ and $a \in U_y$ be arbitrary. Then $U_y = Y \cap U_x$ for some $U_x \in X$. Since $\prt{B}_x$ is a basis for $X$, there is some $B_x \in \prt{B}_x$ such that $a \in B_x \subset U_x$. It follows that $x \in Y \cap B_x \subset Y \cap U_x = U_y$. Since $Y \cap B_x \in \prt{B}_y$, the result follows from Lemma \ref{lemma_collection_is_basis}.
\end{proof}

\setcounter{section}{18}
\section{The Product Topology}

\begin{definition} \label{def_productTopology}
    Let $(X_i)_{i \in I}$ be a collection of topological spaces. The product topology is the set generated by the basis whose elements are 
    \begin{equation*}
        U = \prod_{i\in I} U_i
    \end{equation*} where each $U_i$ is open in $X_i$ and $U_i = X_i$ for all but finitely many $i$.
\end{definition}

\begin{definition} \label{def_metricTopology}
    Let $(X, d)$ be a metric space. The metric topology on $X$ is the topology generated by the basis 
    \begin{equation*}
        \prt{B} = \set{B_{\epsilon}^d(x) : x \in X, \epsilon > 0 \in \R}.
    \end{equation*}
    
    We say that $d$ induces the metric topology on $X$.
\end{definition}

\begin{definition}
    Let $(X, d)$ be a metric space. We define $\overline{d}: X \times X \to \R$ as the metric where $\overline{d}(x, y) = \min{d(x, y), 1}$ for all $x,y \in X$.
\end{definition}


\begin{definition}
   Let $(X_i, d_i)_{i \in I}$ be a collection of metric spaces. The uniform topology on their product $X = \prod_{i \in I} X_i$ is the topology induced by the metric $\overline{d_\infty}: X \times X \to \R$ where $\overline{d_\infty}(x, y) = \sup\set{\overline{d_i}(x_i, y_i) : i \in I}$.
\end{definition}

\begin{theorem} \label{thm_productUniformBoxCmp}
    Let $(X_i, d_i)_{i \in I}$ be a collection of metric spaces. The uniform topology on $X = \prod_{i \in I} X_i$ is finer than the product topology but coarser than the box topology, i.e
    \begin{equation*}
        \prt{T}_{prod} \subset \prt{T}_{unif} \subset \prt{T}_{box}.
    \end{equation*}
\end{theorem}

\begin{proof}
    We first show that $\prt{T}_{prod} \subset \prt{T}_{unif}$, so let $U=\prod_{i \in I} U_i$ be a basis element of the product topology and $(x_{i})_{i \in I} \in U$. Let $\alpha_1, \dots, \alpha_n$ be all the $\alpha$s such that $U_\alpha \neq X_\alpha$. Since $U_{\alpha_j}$ is open in $X_{\alpha_j}$, there is some $\epsilon_j > 0$ such that $B_{\epsilon_j}^{d_j}(x_j) \subset U_{\alpha_j}$. Set $\epsilon = \min{\epsilon_1, \dots, \epsilon_n}$. Then 
    \begin{equation*}
        B_{\epsilon}^{\overline{d_\infty}}(x) \subset \prod_{i \in I} B_{\epsilon}^{d_{i}}(x_i) \subset U,
    \end{equation*} so every set open in the product topology is open in the uniform topology.
    
    Now we show that $\prt{T}_{unif} \subset \prt{T}_{box}$. Let $\ball{\epsilon}{x}{\overline{d}_\infty}$ be a basis element of the uniform topology.
\end{proof}

\subsection*{Exercises}
\begin{exc}{6}
    First, assume that $(x_n) \to x$. Fix some neighborhood $U_\alpha \subset X_\alpha$ and assume for contradiction that we have infinitely many elements in the sequence $(\pi_\alpha(x_n))$ not contained in $U_\alpha$. Then, the set
    \begin{equation*}
        V = \prod_i V_i
    \end{equation*} where 
    \begin{equation*}
        V_i = 
        \begin{cases}
        X_i & i \neq \alpha \\
        U_\alpha & i = \alpha
        \end{cases}
    \end{equation*} is open in the product topology and contains $x$, so only finitely many of the elements in $(x_n)$ are not in $V$. But for each $i$ such that $\pi_\alpha(x_i) \notin U_\alpha$ we have $x_i \notin V$, thus infinitely many $x_i$ are not in $V$, a contradiction.
    
    For the converse direction, assume that $(\pi_\alpha(x_n)) \to \pi_\alpha(x)$ for each $\alpha$ and consider some arbitrary basis element $U = \prod_\alpha U_\alpha$ of the product topology where $x \in U$. Assume for contradiction that we have infinitely many elements of $(x_n)$ not in $U$. Since only finitely many $U_\alpha$'s are not all of $X_\alpha$, there is some $\beta$ such that infinitely many elements of $(\pi_\beta(x_n))$ are not in $U_\beta$. Since $\pi_\beta(x) \in U_\beta$, we have a contradiction.
    
    This fact is not true in general if we use the box topology. Consider the box topology on $\R^\omega = \prod_{n \in \N} \R$, where each $\R$ has the standard topology. Let $(x_n)$ be the sequence where for each $n$ we have
    \begin{equation*}
        x_n = \paren{\frac{n}{n+1},\frac{n+1}{n+2}, \frac{n+2}{n+3}, \dots}.
    \end{equation*} It is easy to see that for each $i \in \N$ the sequence $(\pi_i(x_n))$ indexed by $n$ converges to $\pi_i(x)$, where $x = (1,1,1, \dots)$. Now consider the set 
    \begin{equation*}
        U = \paren{\frac{1}{2}, 2} \times \paren{\frac{2}{3}, 2} \times \paren{\frac{3}{4}, 2} \times \dots
    \end{equation*} which is a neighborhood of $x$ in the box topology.
    
    Notice that $x_1 \notin U$ since $1/2 \notin (1/2, 2)$. Similarly, none of the $x_n$ are in $U$, so the sequence $(x_n)$ does not converge to $x$.
\end{exc}

\begin{exc}{7}
    First we show that the closure of $\R^\infty$ in the box topology is $\R^\infty$. Let $x \in \R^\omega$ be in the closure of $\R^\infty$. This means that any neighborhood $\prod_{i \in \N} U_i$ of $x$ intersects $\R^\infty$, thus all but finitely many $U_i$ must contain zero. Consider the neighborhood 
    \begin{align*}
        &V = \prod_{i \in \N} V_i \\
        &V_i = \begin{cases}
        (0, x_i + 1) & x_i > 0 \\
        (x_i-1, 0) & x_i < 0 \\
        \R & x_i = 0.
        \end{cases}
    \end{align*}
    Clearly we have $x \in V$, so there are only finitely many $V_i$ that do not contain zero, thus $V$ is eventually all of $\R$, but, by the construction of $V$, this can only happen if $x$ is eventually zero. Thus $x \in \R^\infty$, and $\closure{\R^\infty} = \R^\infty$ in the box topology.
    
    Next we show that $\closure{\R^\infty} = \R^\omega$ in the product topology. Let $x \in \R^\omega$ be arbitrary and let $U = \prod_{i \in \N} U_i$ be a neighborhood of $x$. Since $U$ is open in the product topology, every $U_i$ must be all of $\R$ whenever $i \geq I$ for some $I \in \N$. Thus, we have $y = (x_1, \dots, x_{I-1}, 0, 0, 0, \dots) \in \R^\infty$, and $y \in U$. Therefore $U \cap \R^\infty \neq \emptyset$, as we wanted to show.
    
\end{exc}

\chapter{Connectedness and Compactness}

\setcounter{section}{23}
\section{Connected Subspaces of the Real Line}
\subsection*{Exercises}
\begin{exc}{2}
 Let $f : S^1 \to \R$ be a continuous function and let $g : S^1 \to \R$ be a function mapping $x$ to $f(x)-f(-x)$. Notice that $g(x) = 0$ if and only if $f(x)=f(-x)$, and for all $x \in S^1$ we have $g(x) = -g(-x)$. If $g(1, 0) = 0$ then we are done, so assume otherwise. We have either $g(1, 0) > 0 > g(-1, 0)$ or $g(-1, 0) > 0 > g(1, 0)$. In both cases, since $S^1$ is connected and $g$ is continuous, we have some $c \in S^1$ where $g(c) = 0$, by the Intermediate Value Theorem.
\end{exc}

\setcounter{section}{25}
\section{Compact Spaces}
\begin{definition}
   A point $x$ of a topological space $X$ is isolated if and only if the singleton $\set{x}$ is open.
\end{definition}

\begin{lemma} \label{lem_inter_of_compact_nonempty}
    Let $X$ be a compact topological space and $\set{U_i}_{i \in \N}$ be a countable collection of nonempty closed sets with $U_{i+1} \subset U_i$ for every $i \in \N$. Then $\bigcap_{i \in \N} U_i \neq \emptyset$.
\end{lemma}

\begin{proof}
    Assume for contradiction that $\bigcap_{i \in \N} U_i = \emptyset$. It follows by taking the complement on both sides that $\bigcup_{i \in \N} X \setminus{U_i} = X$. Since each $U_i$ is closed their complement is open, so the collection $\set{X \setminus U_i}_{i \in \N}$ is an open cover for $X$, thus it admits a finite subcover $\prt{A} = \set{X \setminus U_{i_1} \dots, X \setminus U_{i_m}}$. It follows that $\bigcap_{j=1}^m U_{i_j} = \emptyset$. Now set $k = \max{i_1, \dots, i_m}$ and choose some $x \in U_k$. Then $x \in U_k \subset U_{k-1} \subset ... \subset U_1$, so $x \in \bigcap_{j=1}^m U_{i_j}$, which is a contradiction.
\end{proof}

\begin{theorem}
    A compact Hausdorff Topological space with no isolated points is uncountable.
\end{theorem}

\begin{proof}
    Let $X$ be a compact Hausdorff topological space with no isolated points. First, we prove the following claim: given any nonempty open $U \subset X$ and any $x \in X$ there is some nonempty open $V \subset U$ such that $x \notin \closure{V}$. Notice that there is some $y \in U$ with $y \neq x$, since if $x \notin U$ we get this by nonemptyness, and if $x \in U$ the result follows since $\set{x}$ cannot be open. By Hausdorfness, there are disjoint open sets $W_1, W_2$ with $x \in W_1$ and $y \in W_2$. Now set $V := W_2 \cap U$. Then $V$ is the set we want, since $V \subset U$ and $x \notin \closure{V}$, as $W_1$ is an open neighborhood of $x$ that does not intersect $V$. Also $V$ is nonempty since $y \in V$.
    
    Now we prove the theorem. Let $f : \N \to X$ be any function. We will show that $f$ is not surjective. Since $X$ is open, there is some open $V_1 \subset X$ where $f(1) \notin \closure{V}$. Similarly, there is some $V_2 \subset V_1$ where $f(2) \notin \closure{V_2}$. We can continue this way to construct a collection of sets so that for every natural number $n$ we have $f(n) \notin \closure{V_n}$ and 
    \begin{equation*}
        \closure{V_1} \supset \closure{V_2} \supset \closure{V_3} \dots
    \end{equation*} with each $V_n$ open and nonempty.
    
    Since $\set{\closure{V_n}}_{n \in \N}$ is a countable collection of nonempty closed sets and $\closure{V_{n+1}} \subset \closure{V_n}$ for each $n \in \N$, Lemma \ref{lem_inter_of_compact_nonempty} implies that there is some $x \in \bigcap_{n \in \N} \closure{V_n}$. But since $x \in V_n$ for every $n \in \N$, we can conclude that $f(n) \neq x$ for every $n \in \N$, so $f$ is not surjective.
\end{proof}

\section{Compact Subspaces of the Real Line}

\begin{definition}
   If $(X, d)$ is a metric space and $A \subset X$ is nonempty, we define $d(x, A) := \inf\set{d(x, y) \mid y \in A}$.
\end{definition}

\begin{definition}
   Let $(X, d)$ be a metric space. If $A \subset X$ is bounded, then the diameter of $A$ is $\sup\set{d(x, y) \mid (x, y) \in A \times A}$.
\end{definition}

\begin{lemma}
    Let $\prt{A}$ be an open cover of the compact metric space $(X, d)$. There exists a $\delta > 0$ such that every subset of $X$ with diameter less than $\delta$ is contained in some element of $\prt{A}$. We call $\delta$ a Lebesgue number of $\prt{A}$.
\end{lemma}

\begin{proof}
    We can assume that no element of $\prt{A}$ is all of $X$. Fix a finite subcover $\set{A_1, \dots, A_n}$ of $\prt{A}$ and define the function 
    \begin{align*}
        &f : X \to \R \\
        &x \mapsto \frac{1}{n} \sum_{i=1}^n d(x, X \setminus A_i).
    \end{align*} Notice that given any  $x \in X$, there is some $A_i$ with $x \in A_i$. By the openness of $A_i$, there is some $r > 0$ with $B_r(x) \subset (A_i)$, so $d(x, X \setminus A_i) \geq r > 0$, thus $f$ is positive at every input. Since $f$ is the sum of continuous functions, $f$ is continuous. Using the compactness of $X$, we know by the Extreme Value Theorem that $f$ attains a minimum $\delta > 0$, so that $f(x) \geq \delta > 0$ for all $x \in X$. We claim that $\delta$ is the Lebesgue number of $\prt{A}$.
    
    First, notice that for every $x \in X$ we have $d(x, X \setminus A_i) \geq \delta$ for some $A_i$, since $f(x) \geq \delta$ and $f$ is the average of all $d(x, X \setminus A_i)$. Now consider any $B \subset A$ with diameter less than $\delta$. For any $x \in B$ we have $x \in B \subset B_\delta(x) \subset A_i$, where $A_i$ is a set with $d(x, X \setminus A_i) \geq \delta$.
\end{proof}

\begin{exc}{2}
    \begin{enumerate}[(a)]
        \item Let $X$ be a subspace of $\R$ in the finite complement topology and let $\prt{A}$ be an open cover for $X$. Given any nonempty $A \in \prt{A}$, $A$ contains all but finitely many points of $X$. For each $x_i \in X$ not contained in $A$ there is some $A_i \in \prt{A}$ which contains $x_i$, since $\prt{A}$ covers $X$. Then the collection $\set{A, A_1, \dots, A_n}$ where $n$ is the amount of points in $X$ not in $A$ is a finite subcover of $\prt{A}$.
        
        \item The subspace $[0, 1] \subset \R$ is not compact when $\R$ is given the countable complement topology. To see this, first fix some bijection $f : \N \to [0, 1] \cap \Q$ and for each $n \in \N$ define $A_n := ([0, 1] \setminus{\Q}) \cup \set{f(n)}$. We claim that $\prt{A} = \set{A_n}_{n \in \N}$ is an open cover for $[0, 1]$.
        
        The complement of each set in $\prt{A}$ is clearly countable, so we only need to check that $\prt{A}$ covers $[0, 1]$. Given any $x \in [0, 1]$, we know that $x \in A_1$ if $x \notin \Q$ and if $x \in \Q$ then $x = f(n)$ for some $n \in \N$, so $x \in A_n$.
        
        Now assume for contradiction that $\prt{A}$ has a finite subcover $\prt{B} = {A_{i_1}, \dots, A_{i_n}}$ and set $k = \max{i_1, \dots, i_n}$. Then $f(k+1) \notin \bigcup_{j=1}^n A_{i_j}$ by the construction of $\prt{A}$, but this contradicts the assumption that $\prt{B}$ covers $[0, 1]$. 
        
    \end{enumerate}
\end{exc}

\part{Algebraic Topology}

\chapter{The Fundamental Group}

We use the convention that every space is topological and every map is continuous.

\section{Basic Constructions}
\subsection{Paths and Homotopy}

\begin{definition}
   A path in $X$ is any map $f : I \to X$. We call $f(0)$ and $f(1)$ the endpoints of $f$. If $f(0)=f(1)$ then $f$ is said to be a loop based at $f(0)$.
\end{definition}

\begin{definition}
   A homotopy of paths is a family of paths $f_t : I \to X$ for each $t \in I$, where there are $x_0, x_1 \in X$ such that $f_t(0) = x_0$ and $f_t(1) = x_1$ for all $t \in I$. We also require that the associated map $F : I \times I \to X$ mapping $(s, t) \mapsto f_t(s)$ is continuous. If $f = f_0$ and $g = f_1$, we say that $f$ is path homotopic to $g$, and write $f \simeq g$.
\end{definition}

\begin{lemma}
    Path homotopy is an equivalence relation.
\end{lemma}

\begin{proof}
    Fix a space $X$ and paths $f, g, h : I \to X$ such that $f(0)=g(0)=h(0)$ and $f(1)=g(1)=h(1)$. Clearly $f \simeq f$ as the family $f_t : I \to S$ mapping $(s, t) \mapsto f_t(s) = f(s)$ is the desired homotopy.
    
    Assume now that $f_t : I \to X$ is a homotopy of paths with $f_0 = f$ and $f_1 = g$. Then $(s, t) \mapsto f_{1-t}(s)$ is a homotopy of paths between $g$ and $f$, thus $g \simeq f$.
    
    Finally, assume that $f \simeq g$ and $g \simeq h$, where $f_t, g_t$ are the relevant homotopies. Then define the homotopy $h_t : I \to X$ as
    
    \begin{equation*}
        h_t(s) = \begin{cases}
        f_{2t}(s), & t \in [0, \frac{1}{2}] \\
        g_{2t-1}(s), & t \in [\frac{1}{2}, 1]
        \end{cases}.
    \end{equation*} This is a path homotopy between $f$ and $h$, thus $f \simeq h$.
\end{proof}

\begin{definition}
   If $f : I \to X$ is a path, then $[f] = \set{g \in X^I \mid f \simeq g}$ is the homotopy class of $f$.
\end{definition}

\begin{definition} \label{def_concatpaths}
   Let $f, g$ be paths where $f(1)=g(0)$. We define the concatenation $f \concat g$ as 
   \begin{equation*}
       (f \concat g)(s) = \begin{cases}
       f(2s), & s \in [0, \frac{1}{2}] \\
       g(2s-1), & s \in [\frac{1}{2}, 1] \\
       \end{cases}.
   \end{equation*} If $f$ and $g$ are loops with the same basepoint, we also define the product $[f][g] = [f \concat g]$.
\end{definition}

\begin{definition}
   Let $x_0 \in X$ be arbitrary. We define the constant loop at $x_0$ as the path $\gamma_0 : I \to X$, where $s \mapsto x_0$. Also, if $f: I \to X$ is a loop around $x_0$, we define the inverse $\bar{f}$ of $f$ as the path $\bar{f} : I \to X$ where $s \mapsto f(1-s)$. 
\end{definition}

\begin{lemma}
    The product in definition \ref{def_concatpaths} is well defined and forms a group with the set $\pi_1(X, x_0) = \set{[f] \mid f(0) = f(1) = x_0}$, where $x_0 \in X$ is some fixed basepoint. The group $\pi_1(X, x_0)$ is called the fundamental group of $X$ at $x_0$.
\end{lemma}

\begin{lemma}
    If $h : I \to X$ is a path with endpoints $x_0$ and $x_1$ respectively, then there is a group isomorphism $\beta_h : \pi_1(X, x_1) \to \pi_1(X, x_0)$. It follows that the fundamental group of a path connected space is unique up to isomorphism.
\end{lemma}

\begin{definition}
   A space is simply connected if and only if it is path connected and its fundamental group is trivial.
\end{definition}

\begin{lemma}
    In a simply connected space, two paths are path homotopic if and only if they share the same endpoints.
\end{lemma}

\begin{proof}
    Assume that $f, g : I \to X$ share the same endpoints, where $X$ is a simply connected space. Then $f \concat \bar{g}$ is a loop at the basepoint $f(0) = x_0$, so $[f \concat \bar{g}] = [\gamma_{x_0}] = [f][\bar{g}]$. Multiplying both sides by $[g]$ we have $[f][\bar{g}][g] = [f] = [g]$, thus $f \simeq g$. The other direction is trivial.
\end{proof}

\begin{definition}
    Let $p : E \to B$ be a surjective map. We say that an open subset $U \subset B$ is evenly covered by $p$ if and only if $p^{-1}(U)$ is a disjoint union of open sets $\set{V_\alpha}$, such that the restriction of $p$ to each $V_\alpha$ is a homeomorphism onto $U$. If every $x \in B$ has an open neighborhood that is evenly covered by $p$, we say that $p$ is a covering map, and $E$ is a covering space of $B$.
\end{definition}

\begin{definition}
    Let $p : E \to B$ be a map and $f : I \to B$ be a path. If $\tilde{f} : I \to  E$ is such that $f = p \circ \tilde{f}$, we say that $\tilde{f}$ is a lifting of $f$.
\end{definition}

\begin{lemma} \label{lem_lift_path}
    Let $p : E \to B$ be a covering map. For all paths $f : I \to B$ beginning at $b_0$ and all $e_0 \in p^{-1}(b_0)$, there is a unique path $\tilde{f} : I \to E$ that begins at $e_0$ and lifts $f$.
\end{lemma}

\begin{proof}
    Let $f : I \to B$ be a path beginning at $b_0$ and fix some $e_0 \in p^{-1}(b_0)$. Choose some open covering of $B$ by sets $U$ that are evenly covered by $p$. Using the Lebesgue number lemma, we can find some subdivision $s_0, \dots, s_n$ of $[0, 1]$ such that for each $[s_i, s_{i+1}]$ we have $f([s_i, s_{i+1}]) \subset U$ for some $U$ in the covering we fixed.
    
    We define $\tilde{f} : I \to E$ inductively. First, set $\tilde{f}(0) = e_0$. Now assume that $\tilde{f}$ is defined for all $s$ with $0 \leq s \leq s_i$. We know that $f([s_i, s_{i+1}])$ lies on an an open $U$, where its preimage $p^{-1}(U)$ is a disjoint union of open sets $\set{V_\alpha}$ that are mapped homeomorphically onto $U$ by $p$. Then $\tilde{f}(s_i)$ is in one of those sets, say $V_0$. Then, for $s \in [s_i, s_{i+1}]$, we define $\tilde{f}(s) = (p\restriction{V_0})^{-1}(f(s))$. By the pasting lemma, $\tilde{f}$ is continuous.
    
    To see that $\tilde{f}$ is unique, let $\tilde{g} : I \to E$ be another lift of $f$ starting at $e_0$. We use induction again to show that $\tilde{f} = \tilde{g}$. We have $\tilde{f}(0) = e_0 = \tilde{g}(0)$, so assume that $\tilde{f} = \tilde{g}$ on the interval $[0, s_i]$. Let $V_0$ be as in the preceding paragraph. Since $\tilde{g}$ it must carry $[s_i, s_{i+1}]$ to $p^{-1}(U) = \bigcup_\alpha V_\alpha$. So $\tilde{f}(s_i) = \tilde{g}(s_i) \in V_\alpha$, but the $V_\alpha$ are disjoint, thus $\tilde{g}(s_i) \in V_0$. Since $\tilde{g}([s_i, s_{i+1}])$ is connected, we must also have $\tilde{g}([s_i, s_{i+1}]) \subset V_0$. Then, given any $s \in [s_i, s_{i+1}]$, we must have $f(s) = p(\tilde{g}(s))$, so $\tilde{g}(s)$ must be a point in $V_0$ that is also in the preimage of $f(s)$, but there is only one such point, namely $\tilde{f}(s)$.
\end{proof}

\begin{lemma} \label{lem_lift_homotopy}
    Let $p : E \to B$ be a covering map and $F : I \times I \to B$ be a continuous map with $F(0, 0) = b_0$. For each $e_0 \in p^{-1}(b_0)$ there is a unique continuous lifting $\tilde{F} : I \times I \to E$ of $F$ such that $\tilde{F}(0, 0) = e_0$. Furthermore, if $F$ is a path homotopy, then so is $\tilde{F}$.
\end{lemma}

\begin{proof}
    We can uniquely construct $\tilde{F}$ analogously to the construction in Lemma \ref{lem_lift_path}, instead dividing the rectangle $I \times I$ into small rectangles and defining $\tilde{F}$ inductively on these rectangles.
    
    Now assume that $F$ is a path homotopy. Then $F$ maps $0 \times I$ to the point $b_0$, so we must have $\tilde{F}(0 \times I) = p^{-1}(\set{b_0})$. But $p^{-1}(\set{b_0})$ has the discrete topology, so its subspaces that contain two points are disconnected. But $\tilde{F}$ is continuous and $0 \times I$ is connected, so $\tilde{F}(0 \times I)$ is connected, hence it is a one point set. Similarly, $\tilde{F}(1 \times I)$ is also a one point set, so $\tilde{F}$ is a path homotopy.
\end{proof}

\begin{theorem} \label{thm_hom_lift_hom}
    Let $p : E \to B$ be a covering map and $f, g : I \to B$ be paths beginning at $b_0$ that are path homotopic. It follows that any liftings $\tilde{f}, \tilde{g} : I \to E$ which begin at $e_0 \in p^{-1}(b_0)$ are path homotopic.
\end{theorem}

\begin{proof}
    Let $F : I \times I \to B$ be a path homotopy between $f$ and $g$. The lifting $\tilde{F} : I \times I \to E$ beginning at $e_0$ is also a path homotopy. Then $\tilde{F}\restriction{I \times 0}$ is a lifting of $f$ beginning at $e_0$, so $\tilde{F}\restriction{I \times 0} = \tilde{f}$, by the uniqueness part of Lemma \ref{lem_lift_path}. Similarly, $\tilde{F}\restriction{I \times 1} = \tilde{g}$. Thus $\tilde{F}$ is a path homotopy between $\tilde{f}$ and $\tilde{g}$. 
\end{proof}

\begin{definition}
    Let $p : E \to B$ be a covering map. Fix some $b_0 \in B$ and $e_0 \in p^{-1}(\set{b_0})$. Let $\Phi : \pi_1(B, b_0) \to p^{-1}(\set{b_0})$ be the mapping taking $[f]$ to $\tilde{f}(1)$, where $\tilde{f} : I \to E$ is the unique lift of $f$ starting at $e_0$. This map is well defined by Theorem \ref{thm_hom_lift_hom}, and we call $\Phi$ a lifting correspondence between $\pi_1(B, b_0)$.
\end{definition}

\begin{lemma}
    Let $p : E \to B$ be a covering map and $p(e_0) = b_0$. If $E$ is path connected, then the lifting correspondence $\Phi : \pi_1(B, b_0) \to p^{-1}(\set{b_0})$ is surjective, and if $E$ is simply connected then $\Phi$ is bijective.
\end{lemma}

\begin{proof}
    Assume first that $E$ is path connected and fix some $e \in p^{-1}(\set{b_0})$. Let $\tilde{f} : I \to E$ be a path connecting $e_0 \to e$ and define $f = p \circ \tilde{f}$. Then $\tilde{f}$ is a lifting of $f$, which is a path beginning at $b_0$. It follows that $\Phi([f]) = \tilde{f}(1) = e$, so $\Phi$ is surjective.
    
    Now assume that $E$ is simply connected and that $\Phi([f]) = \Phi([g])$. Let $\tilde{f}, \tilde{g} : I \to E$ be the lifts of $f$ and $g$ which begin at $e_0$. It follows that $\tilde{f}(1) = \tilde{g}(1)$, hence $\tilde{f}$ and $\tilde{g}$ share the same endpoints and thus are homotopic. Let $\tilde{F} : I \times I \to E$ be a homotopy between them. The map $F = p \circ \tilde{F}$ is a homotopy between $f$ and $g$, so $[f] = [g]$, and $\Phi$ is injective.
\end{proof}

\begin{lemma}
    The function 
    \begin{align*}
        p : \R &\to S^1 \\
        s &\mapsto (\cos{2 \pi s}, \sin{2 \pi s})
    \end{align*} is a covering map.
\end{lemma}

\begin{theorem}
    $\pi_1(S^1)$ is isomorphic to the additive group $\Z$.
\end{theorem}

\begin{proof}
    Let $p$ be the covering map given in the previous lemma. Let $\Phi : \pi_1(S^1, (1, 0)) \to p^{-1}(\set{0})$ be a lifting correspondence, and notice that $p^{-1}(\set{0}) = \Z$. We know that $\R$ is simply connected which implies that $\Phi$ is bijective, hence we only need to show that $\Phi$ is a group homomorphism.
    
    Choose arbitrary loops $[f], [g] \in \pi_1(S^1, (1,0))$ and let $\tilde{f}, \tilde{g}$ be the liftings of $f$ and $g$ beginning at $0$. Notice that the right endpoints $\tilde{f}(1) = n$ and $\tilde{g}(1) = m$ are integers. The function $\tilde{g}' : I \to \R$ given by $\tilde{g}'(s) = g(s) + n$ is a lift of $g$ beginning at $n$, since $p$ has period $1$. Then we can concatenate $\tilde{f}$ and $\tilde{g}'$, and $\tilde{f} \concat \tilde{g}'$ is the unique lift of $f \concat g$ which begins at $0$. Now using the fact that $(\tilde{f} \concat \tilde{g}')(1) = n + m$, we have 
    \begin{equation*}
        \Phi([f] \concat [g]) = \Phi([f \concat g]) = (\tilde{f} \concat \tilde{g}')(1) = n + m = \Phi([f]) + \Phi([g]).
    \end{equation*} Since $\Phi$ is a bijective homomorphism, it is also a group isomorphism between $\pi_1(S^1)$ and the additive group of the integers.
\end{proof}

\begin{remark}
     For each $n \in \Z$, let $\tilde{\omega}_n : I \to \R$ be given by $\tilde{\omega}_n(s) = n s$. Then $\tilde{\omega}_n$ is a path from $0$ to $n$. Define also $\omega_n = p \circ \tilde{\omega}_n$. We have $\Phi([f]) = n \iff [f] = [\omega_n]$, that is, the group isomorphism $\Phi$ maps homotopy classes of paths that loop $n$ times around $S^1$ to the integer $n$.
     
     To see this, notice that $\tilde{\omega}_n$ is a lift of $\omega_n$ beginning at $0$ and ending at $n$, so $\Phi([\omega_n]) = n$. Thus $\Phi([f]) = n \iff \Phi([f]) = \Phi([\omega_n]) \iff [f] = [\omega_n]$, since $\Phi$ is injective.
\end{remark}

\begin{corollary}[The Fundamental Theorem of Algebra]
    Every nonconstant polynomial with coefficients in $\C$ has a root in $\C$.
\end{corollary}

\begin{proof}
    Let $p : \C \to \C$ be a polynomial with no roots given by $p(z) = z^n + a_1 z^{n-1} + \dots + a_n$. For each $r \geq 0$ define the path 
    \begin{align*}
        f_r : I &\to S^1 \\
        s &\mapsto \frac{p(r e^{2 \pi i s})/p(r)}{p(r e^{2 \pi i s})/p(r)}. 
    \end{align*} Since $p$ has no roots, $f_r$ is a continuous loop around $S^1$ based at $1$. Then, $f_0 = \omega_0$ which is the constant loop based at $1$, so each $f_r$ is path homotopic to $\omega_0$. Now choose some $R \in \R$ so that $R > \max{1, |a_1| + \dots + |a_n|}$. When $|z| = R$, we have $|z^n| > |z^{n-1}|(|a_1| + \dots + |a_n|) \geq |a_1 z^{n-1} + \dots + a_n|$, by the triangle inequality and the fact that $R^p > R^q$ when $p > q$, since $R > 1$.
    
    By the previous inequality, it follows that the function $p_t : \C \to S^1$ given by $p_t(z) = z^n + t(a_1 z^{n-1} + \dots + |a_n|)$ has no roots for $t \in [0, 1]$. Now define the map 
    \begin{align*}
        g_t : I &\to S^1 \\
        s &\mapsto \frac{p_t(R e^{2 \pi i s})/p_t(R)}{p_t(R e^{2 \pi i s})/p_t(R)}. 
    \end{align*} Notice that $p_0 = z \mapsto z^n$ and $p_1 = p$, which means $g_0 = s \mapsto e^{2 \pi i n s} = \omega_n$, and $g_1 = f_R$. But clearly $g_0 \simeq g_1$, and we know that $f_R \simeq f_0 \simeq \omega_0$, so $[\omega_n] = [\omega_0]$. But then $\Phi([\omega_n]) = n = \Phi([\omega_0]) = 0$. Hence $n = 0$, and our polynomial is constant.
\end{proof}

\begin{corollary}
    Every continuous map $h : D^2 \to D^2$ has a fixed point, that is, some $x \in D^2$ with $h(x) = x$.
\end{corollary}

\begin{proof}
    Let $h : D^2 \to D^2$ be continuous and assume for contradiction that it has no fixed points. Define the map $r : D^2 \to S^1$ in the following way. For each $x \in D^2$ consider the ray beginning at $x$ which passes through $h(x)$. Let $r(x)$ be the intersection of this ray with the circle. $r$ is a continuous map, and its restriction to $S^1$ is the identity.
    
    Now consider the loop $\omega_1 : I \to S^1$. In $D^2$, there is a homotopy $f_t : I \to D^2$ from $\omega_1$ to a constant loop, given by $f_t(s) = (1-t) \omega_1(s)$. Then, $r \circ f_t$ is a homotopy in $S^1$ is a homotopy of paths from $\omega_1$ to a constant loop based at $0$. Thus $[\omega_1] = [\omega_0]$ and we have $\Phi([\omega_1]) = 1 = \Phi([\omega_0]) = 0$, a contradiction.
\end{proof}

\end{document}