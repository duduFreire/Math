\documentclass{article}
\usepackage[utf8]{inputenc}
\usepackage[english]{babel}
\usepackage{amsmath}
\usepackage{amsthm}
\usepackage{mathtools}
\usepackage{amsfonts}
\usepackage{indentfirst}
\usepackage{enumitem}
\usepackage{microtype}
\usepackage[colorlinks=true,linkcolor=blue]{hyperref}

\setlist{parsep=0pt,listparindent=\parindent}

\newtheorem{theorem}{Theorem}[subsection]
\newtheorem{lemma}{Lemma}[subsection]

\newtheorem{corollary}{Corollary}[theorem]
\newtheorem{proposition}{Proposition}[subsection]
\theoremstyle{definition}
\newtheorem{definition}{Definition}[subsection]
\theoremstyle{remark}
\newtheorem{remark}{Remark}[subsection]

\newenvironment{exc}[1]{\noindent\textbf{Exercise \thesubsection.#1.}}{\medskip}


\DeclarePairedDelimiter\abs{\lvert}{\rvert}
\DeclarePairedDelimiter\ceil{\lceil}{\rceil}
\DeclarePairedDelimiter\floor{\lfloor}{\rfloor}

\makeatletter
\let\oldabs\abs
\let\oldceil\ceil 
\let\oldfloor\floor
\def\abs{\@ifstar{\oldabs}{\oldabs*}}
\def\ceil{\@ifstar{\oldceil}{\oldceil*}}
\def\floor{\@ifstar{\oldfloor}{\oldfloor*}}
\makeatother

\newcommand{\N}{\mathbb{N}}
\newcommand{\Z}{\mathbb{Z}}
\newcommand{\Q}{\mathbb{Q}}
\newcommand{\I}{\mathbb{I}}
\newcommand{\R}{\mathbb{R}}
\newcommand{\paren}[1]{\left( #1 \right)}
\newcommand{\set}[1]{\{#1\}}
\newcommand{\prt}[1]{\mathcal{#1}}
\newcommand{\lep}[1][L]{#1et $\epsilon > 0$ be arbitrary}
\newcommand{\seq}[1]{\left(#1\right)_{n=1}^\infty}
\newcommand{\LIM}[1]{\text{LIM}_{n \to \infty} #1}
\newcommand{\ball}[3]{B_{#1}^{#3}(#2)}
\newcommand{\closure}[2][3]{%
{}\mkern#1mu\overline{\mkern-#1mu#2}}

\let\oldlog\log
\let\oldmax\max
\let\oldmin\min
\let\oldsin\sin
\let\oldcos\cos
\renewcommand{\log}[1]{\oldlog \left( #1 \right)}
\renewcommand{\max}[1]{\oldmax \left( #1 \right)}
\renewcommand{\min}[1]{\oldmin \left( #1 \right)}
\renewcommand{\sin}[1]{\oldsin \left( #1 \right)}
\renewcommand{\cos}[1]{\oldcos \left( #1 \right)}


\title{Topology}
\author{Eduardo Freire}
\date{August 2021}

\begin{document}

\maketitle

\setcounter{section}{1}
\section{Topological Spaces and Continuous Functions}

\setcounter{subsection}{11}
\subsection{Topological Spaces}


\begin{definition}\label{def_topology}
    A topology $\prt{T}$ on a set $X$ is a collection of subsets of $X$ satisfying the following conditions: 
    
    \begin{enumerate}
        \item $\emptyset, X \in \prt{T}$,
        \item If $U_\lambda \in \prt{T}$ for every $\lambda \in \Lambda$, then $\paren{\bigcup_{\lambda \in \Lambda} U_\lambda} \in \prt{T}$ and
        \item If $A, B \in \prt{T}$, then $A \cap B \in \prt{T}$.
    \end{enumerate}
    A subset $U$ of $X$ is called open if and only if $U \in \prt{T}$.
\end{definition}

\begin{definition}
    Let $\prt{T}, \prt{T'}$ be topologies on $X$. We say that $\prt{T'}$ is finer than $\prt{T}$ if and only if $\prt{T} \subset \prt{T'}$. Similarly, $\prt{T'}$ is coarser than $\prt{T}$ if and only if $\prt{T'} \subset \prt{T}$. 
\end{definition}

\subsection{Basis for a Topology}

\begin{definition}\label{def_basis}
    A collection $\prt{B}$ of subsets of $X$ is called a basis for a topology on $X$ if and only if it satisfies the following conditions:
    \begin{enumerate}
        \item $X = \bigcup_{B \in \prt{B}} B$ and
        \item If $B_1, B_2 \in \prt{B}$ and $x \in B_1 \cap B_2$, then there is some $B_3 \in \prt{B}$ such that $x \in B_3$ and $B_3 \subset B_1 \cap B_2$.
    \end{enumerate}
    Given a basis $\prt{B}$ on a set $X$, let $\prt{T}$ be the set such that $U \in \prt{T}$ if and only if for every $x \in U$ there is some $B_x \in \prt{B}$ such that $x \in B_x$ and $B_x \subset U$. We call $\prt{T}$ the set generated by $\prt{B}$.
\end{definition}

\begin{proposition}
    Let $X$ be a set and $\prt{B}$ be a a basis for $X$. The set $\prt{T}$ generated by $\prt{B}$ is a topology on $X$.
\end{proposition}

\begin{proof}
    It is easy to see that clauses $1$ and $2$ in Definition \ref{def_topology} hold using the first clause in the definition of a basis. For the last clause, assume that $A,B \in \prt{T}$ and let $x \in A \cap B$ be arbitrary. Since $A$ and $B$ are in $\prt{T}$, there are $B_1, B_2 \in \prt{B}$ such that $x \in B_1 \subset A$ and $x \in B_2 \subset B$. It follows that $x \in B_1 \cap B_2 \subset A \cap B$. By clause 2 in the definition of a basis, there is some $B_3 \in \prt{B}$ such that $x \in B_3 \subset B_1 \cap B_2 \subset A \cap B$, thus $A \cap B \in \prt{T}$.
\end{proof}

\newpage
\begin{lemma} \label{lemma_basis_makes_topology}
    Let $\prt{B}$ be the basis for a topology $\prt{T}$ on $X$ (so $\prt{T}$ is the topology generated by $\prt{B}$). Then $\prt{T}$ is the collection of all unions of elements of $\prt{B}$.
\end{lemma}

\begin{proof}
    Let $U \in \prt{T}$ be arbitrary. We wish to show that there is some collection of elements in $\prt{B}$ such that their union is $U$. By Definition \ref{def_basis}, for each $x \in U$ we can choose some $B_x \in \prt{B}$ such that $x \in B_x \subset U$. It is straightforward to see that $\bigcup_{x \in U} B_x = U$. Also, since the elements of $\prt{B}$ are subsets of $X$, it is evident that their union is a subset of $X$, and the result follows.
\end{proof}

\begin{lemma} \label{lemma_finer_by_basis}
    Let $\prt{B}, \prt{B'}$ be basis for the topologies $\prt{T}, \prt{T'}$ respectively on a set $X$. Then the following are equivalent:
    
    \begin{enumerate}
        \item $\prt{T'}$ is finer than $\prt{T}$,
        \item For every $B \in \prt{B}$ and every $x \in B$, there is some $B' \in \prt{B'}$ such that $x \in B' \subset B$.
    \end{enumerate}
\end{lemma}

\begin{proof}
    For the forward direction assume (1), i.e that $\prt{T} \subset \prt{T'}$. Let $B \in \prt{B}$ and $x \in B$ be arbitrary. It is easy to see that every element of a basis for a topology is an open set in that topology, specifically $B \in \prt{T}$. Thus $B \in \prt{T'}$, and definition \ref{def_basis} guarantees that there is some $B'_x \in \prt{B'}$ such that $x \in B' \subset B$, as we wanted to show.
    
    Now assume clause number (2) and let $U \in \prt{T}$ be arbitrary. We need to show that $U \in \prt{T'}$, so let $x \in U$ be arbitrary. We know, since $\prt{T}$ is generated by $\prt{B}$, that there is some $B \in \prt{B}$ such that $x \in B \subset U$. By (2), there is also some $B' \in \prt{B'}$ such that $x \in B' \subset B \subset U$. Since $\prt{T'}$ is generated by $\prt{B'}$, this means $U \in \prt{T'}$, as we wanted to show.
\end{proof}


\begin{lemma} \label{lemma_collection_is_basis}
    Let $X$ be a set and $\prt{T}$ be a topology on $X$. If $\prt{C}$ is a collection of open sets of $X$ such that for every $U \in \prt{T}$ and every $x \in U$ there is some $C \in \prt{C}$ such that $x \in C \subset U$, then $\prt{C}$ is a basis on $X$. Furthermore, the topology generated by $\prt{C}$ is $\prt{T}$.
\end{lemma}

\begin{proof}
    Assume the hypothesis in the lemma. To show that $\prt{C}$ meets clause (1) of definition \ref{def_basis}, we need to show that for any given $x \in X$ there is some $C \in \prt{C}$ such that $x \in C$, so let $x$ be arbitrary. We now that $X$ is open, so the hypothesis of the lemma guarantees that there is some $c \in \prt{C}$ with $x \in C$.
    
    Next, assume that $C_1, C_2 \in \prt{C}$ and $x \in C_1 \cap C_2$. Since $C_1, C_2$ are open, their intersection must also be open. By the lemma hypothesis, there is some $C_3 \in \prt{C}$ such that $x \in C_3 \subset C_1 \cap C_2$, so clause (2) of definition \ref{def_basis} is met and $\prt{C}$ is a basis for $\prt{T}$.
    
    Now let the collection of subsets $\prt{T'}$ be such that $U' \in \prt{T'}$ if and only if for every $x \in U'$ there is some $C_x \in \prt{C}$ such that $x \in C_x \subset U'$. We need to show that $\prt{T} = \prt{T'}$. Assume first that $U \in \prt{T}$. The lemma hypothesis guarantees that for any $x \in U$ there is some $C \in \prt{C}$ such that $x \in C \subset U$, thus $U \in \prt{T'}$. By Lemma \ref{lemma_basis_makes_topology}, $\prt{T'}$ is the collection of all unions of elements of $\prt{C}$. So given some $U' \in \prt{T'}$, $\prt{T'}$ is some arbitrary union of elements in $\prt{C}$, but every $C \in \prt{C}$ is open, so their union is also open. This means that $U' \in \prt{T'}$, thus $\prt{T} = \prt{T'}$.
\end{proof}

\begin{definition}\label{def_subbasis}
    A subbasis $\prt{S}$ for a topology on $X$ is a collection of subsets of $X$ such that for every $x \in X$ there is some $S \in \prt{S}$ such that $x \in S$. The topology generated by $\prt{S}$ is collection of all the arbitrary unions of finite intersections of elements of $\prt{S}$.
\end{definition}

\begin{remark}
     It might not be clear at first that the set generated by $\prt{S}$ is a topology on $X$. To see that it is, notice that the collection of all finite intersections of elements of $\prt{S}$ is a basis $\prt{B}$. Then, the collection of all arbitrary unions of elements of $\prt{B}$ is the topology generated by $\prt{B}$, according to Lemma \ref{lemma_basis_makes_topology}.
\end{remark}

\subsection{The Order Topology}

\begin{definition}\label{def_order_basis}
    Let $X$ be a set with more than one element and $<$ be a strict linear order on $X$. We define the set $\prt{B}$ by 
    \begin{align*}
        \prt{B} := &\set{(x, y) : x < y} \cup \\ &\set{[x_0, y) : x_0 < y\text{, if $X$ has a least element $x_0$.}} \cup \\ 
        &\set{(x, y_0] : x < y_0\text{, if $X$ has a largest element $y_0$.}}
    \end{align*}
    We call $\prt{B}$ the order basis on $X$ with order $<$, and the topology it generates is called the order topology.
\end{definition}

\begin{proposition}
    Given any $X$ with more than one element and some strict linear order $<$ on $X$, the order basis $\prt{B}$ is a basis for a topology on $X$.
\end{proposition}

\begin{proof}
    Let $x \in X$ be arbitrary. We know that there is some $y \in X$ other than $x$. If $x < y$ and $x$ is the least element of $x$, then $x \in [x, y) \in \prt{B}$, otherwise there is some $z \in X$ such that $z < x < y$, thus $x \in (z, y) \in \prt{B}$. Similarly, we can show that when $y < x$ there is some $B$ such that $x \in B \in \prt{B}$.
    
    Now let $B_1, B_2 \in \prt{B}$ and $x \in B_1 \cap B_2$ be arbitrary. It is straightforward but tedious to check that there is some $B_3 \in \prt{B}$ such that $x \in B_3 \subset B_1 \cap B_2$.
\end{proof}

\begin{definition}
    The standard topology on $\R$ is the one generated by the basis $\prt{B} = \set{(a, b) : a < b}$.

    The lower limit topology $\R_{\prt{L}}$ on $\R$ is the topology generated by the basis $\prt{B'} = \set{[a, b) : a < b}$.
\end{definition}

\begin{lemma}
    The lower limit topology on the reals is strictly finer then the standard topology.
\end{lemma}

\begin{proof}
    To show that $\R_{\prt{L}}$ is finer than the standard topology, it suffices to show that given any $B$ in the standard basis $\prt{B}$ and any $x \in \R$, there is some $B' \in \prt{B'}$ such that $x \in B' \subset B$, by Lemma \ref{lemma_finer_by_basis}. So let $B \in \prt{B}$ and $x \in \R$ be arbitrary. We know that $B = (a, b)$ with $a < b$ and $a < x < b$. Then $x \in [x, b) \in \prt{B'}$ and $[x, b) \subset B$, as we wanted to show.
    
    Also, the interval $[0, 1)$ is open in the lower limit topology, but not in the standard. To see that, assume for contradiction that $[0, 1)$ is open in the standard topology. Then, there must be some $(a, b) \in \prt{B}$ such that $0 \in (a, b) \subset [0, 1)$. Since $0 \in (a, b)$, $a < 0 < b$. Then $a < a/2 < 0 < b$, so $a/2 \in (a, b)$, therefore $a/2 \in [0, b)$. Thus $a/2 \geq 0$, a contradiction. Thus, $\R_{\prt{L}}$ is strictly finer than the standard topology.
\end{proof}

\subsection{The Product Topology on \texorpdfstring{$X \times Y$}{X x Y}}

\begin{definition}
    Let $X$ and $Y$ be topological spaces. The product topology $X \times Y$ is defined as the topology generated by the basis $\prt{B} = \set{U \times V : \text{$U$ is open in $X$ and $V$ is open in $Y$}}.$
\end{definition}

\begin{lemma}
    Let $\prt{B}_x, \prt{B}_y$ be basis for $X$ and $Y$ respectively. It follows that $\prt{B}_x \times \prt{B}_y$ generates the product topology $X \times Y$.
\end{lemma}

\begin{proof}
    We apply Lemma \ref{lemma_collection_is_basis} to the collection $\prt{B}_x \times \prt{B}_y$ of open sets. Let $W$ be open in $X \times Y$ and $a \times b \in W$ be arbitrary. By the definition of the order topology, there is some $B \in \prt{B}$ such that $a \times b \in U \times V \subset W$, where $\prt{B}$ is the basis for $X \times Y$. Since $\prt{B}_x$ is a basis for $X$, there is some $B_x \in \prt{B}_x$ such that $a \in B_x \subset U$. Similarly, there is some $B_y \in \prt{B}_y$ such that $b \in B_y \subset V$. Then $a \times b \in B_x \times B_y \subset U \times V \subset W$, thus the conditions of the lemma just mentioned are met and $\prt{B}_x \times \prt{B}_y$ is a basis and generates the product topology.
\end{proof}

\subsection{The Subspace Topology}

\begin{definition}\label{def_subspace_topology}
    Let $(X, \prt{T}_x)$ be a topological space. For any $Y \subset X$, we define the subspace topology as $\prt{T}_y = \set{Y \cap U : U \in \prt{T}_x}$.
\end{definition}

\begin{lemma}
    The set constructed in definition \ref{def_subspace_topology} is a topology on $X$.
\end{lemma}

\begin{proof}
    By definition, $X \in \prt{T}_x$, so $Y \cap X = Y \in \prt{T}_y$, and similarly for the empty set. Now let $\set{Y \cap U_\lambda : \lambda \in \Lambda}$ be a collection of open sets in $\prt{T}_y$. Then 
    \begin{equation*}
        \bigcup_{\lambda \in \Lambda} Y \cap U_{\lambda} = \paren{Y \cap  \bigcup_{\lambda \in \Lambda} U_{\lambda}} \in \prt{T}_y, 
    \end{equation*} since arbitrary union of sets in $\prt{T}_x$ are open. A similar argument shows that finite intersections of sets in $\prt{T}_y$ are also in $\prt{T}_y$.
\end{proof}

\begin{lemma}
    Let $X$ be a topological space and $Y$ be the subspace topology on $X$ generated by $Y \subset X$. If $\prt{B}_x$ is a basis for $X$ then $\prt{B}_y = \set{Y \cap B_x : B_x \in \prt{B}_x}$ is a basis for $Y$.
\end{lemma}

\begin{proof}
    Let $U_y \in \prt{T}_y$ and $a \in U_y$ be arbitrary. Then $U_y = Y \cap U_x$ for some $U_x \in X$. Since $\prt{B}_x$ is a basis for $X$, there is some $B_x \in \prt{B}_x$ such that $a \in B_x \subset U_x$. It follows that $x \in Y \cap B_x \subset Y \cap U_x = U_y$. Since $Y \cap B_x \in \prt{B}_y$, the result follows from Lemma \ref{lemma_collection_is_basis}.
\end{proof}

\setcounter{subsection}{18}
\subsection{The Product Topology}

\begin{definition} \label{def_productTopology}
    Let $(X_i)_{i \in I}$ be a collection of topological spaces. The product topology is the set generated by the basis whose elements are 
    \begin{equation*}
        U = \prod_{i\in I} U_i
    \end{equation*} where each $U_i$ is open in $X_i$ and $U_i = X_i$ for all but finitely many $i$.
\end{definition}

\begin{definition} \label{def_metricTopology}
    Let $(X, d)$ be a metric space. The metric topology on $X$ is the topology generated by the basis 
    \begin{equation*}
        \prt{B} = \set{B_{\epsilon}^d(x) : x \in X, \epsilon > 0 \in \R}.
    \end{equation*}
    
    We say that $d$ induces the metric topology on $X$.
\end{definition}

\begin{definition}
    Let $(X, d)$ be a metric space. We define $\overline{d}: X \times X \to \R$ as the metric where $\overline{d}(x, y) = \min{d(x, y), 1}$ for all $x,y \in X$.
\end{definition}


\begin{definition}
   Let $(X_i, d_i)_{i \in I}$ be a collection of metric spaces. The uniform topology on their product $X = \prod_{i \in I} X_i$ is the topology induced by the metric $\overline{d_\infty}: X \times X \to \R$ where $\overline{d_\infty}(x, y) = \sup\set{\overline{d_i}(x_i, y_i) : i \in I}$.
\end{definition}

\begin{theorem} \label{thm_productUniformBoxCmp}
    Let $(X_i, d_i)_{i \in I}$ be a collection of metric spaces. The uniform topology on $X = \prod_{i \in I} X_i$ is finer than the product topology but coarser than the box topology, i.e
    \begin{equation*}
        \prt{T}_{prod} \subset \prt{T}_{unif} \subset \prt{T}_{box}.
    \end{equation*}
\end{theorem}

\begin{proof}
    We first show that $\prt{T}_{prod} \subset \prt{T}_{unif}$, so let $U=\prod_{i \in I} U_i$ be a basis element of the product topology and $(x_{i})_{i \in I} \in U$. Let $\alpha_1, \dots, \alpha_n$ be all the $\alpha$s such that $U_\alpha \neq X_\alpha$. Since $U_{\alpha_j}$ is open in $X_{\alpha_j}$, there is some $\epsilon_j > 0$ such that $B_{\epsilon_j}^{d_j}(x_j) \subset U_{\alpha_j}$. Set $\epsilon = \min{\epsilon_1, \dots, \epsilon_n}$. Then 
    \begin{equation*}
        B_{\epsilon}^{\overline{d_\infty}}(x) \subset \prod_{i \in I} B_{\epsilon}^{d_{i}}(x_i) \subset U,
    \end{equation*} so every set open in the product topology is open in the uniform topology.
    
    Now we show that $\prt{T}_{unif} \subset \prt{T}_{box}$. Let $\ball{\epsilon}{x}{\overline{d}_\infty}$ be a basis element of the uniform topology.
\end{proof}

\begin{exc}{6}
    First, assume that $(x_n) \to x$. Fix some neighborhood $U_\alpha \subset X_\alpha$ and assume for contradiction that we have infinitely many elements in the sequence $(\pi_\alpha(x_n))$ not contained in $U_\alpha$. Then, the set
    \begin{equation*}
        V = \prod_i V_i
    \end{equation*} where 
    \begin{equation*}
        V_i = 
        \begin{cases}
        X_i & i \neq \alpha \\
        U_\alpha & i = \alpha
        \end{cases}
    \end{equation*} is open in the product topology and contains $x$, so only finitely many of the elements in $(x_n)$ are not in $V$. But for each $i$ such that $\pi_\alpha(x_i) \notin U_\alpha$ we have $x_i \notin V$, thus infinitely many $x_i$ are not in $V$, a contradiction.
    
    For the converse direction, assume that $(\pi_\alpha(x_n)) \to \pi_\alpha(x)$ for each $\alpha$ and consider some arbitrary basis element $U = \prod_\alpha U_\alpha$ of the product topology where $x \in U$. Assume for contradiction that we have infinitely many elements of $(x_n)$ not in $U$. Since only finitely many $U_\alpha$'s are not all of $X_\alpha$, there is some $\beta$ such that infinitely many elements of $(\pi_\beta(x_n))$ are not in $U_\beta$. Since $\pi_\beta(x) \in U_\beta$, we have a contradiction.
    
    This fact is not true in general if we use the box topology. Consider the box topology on $\R^\omega = \prod_{n \in \N} \R$, where each $\R$ has the standard topology. Let $(x_n)$ be the sequence where for each $n$ we have
    \begin{equation*}
        x_n = \paren{\frac{n}{n+1},\frac{n+1}{n+2}, \frac{n+2}{n+3}, \dots}.
    \end{equation*} It is easy to see that for each $i \in \N$ the sequence $(\pi_i(x_n))$ indexed by $n$ converges to $\pi_i(x)$, where $x = (1,1,1, \dots)$. Now consider the set 
    \begin{equation*}
        U = \paren{\frac{1}{2}, 2} \times \paren{\frac{2}{3}, 2} \times \paren{\frac{3}{4}, 2} \times \dots
    \end{equation*} which is a neighborhood of $x$ in the box topology.
    
    Notice that $x_1 \notin U$ since $1/2 \notin (1/2, 2)$. Similarly, none of the $x_n$ are in $U$, so the sequence $(x_n)$ does not converge to $x$.
\end{exc}

\begin{exc}{7}
    First we show that the closure of $\R^\infty$ in the box topology is $\R^\infty$. Let $x \in \R^\omega$ be in the closure of $\R^\infty$. This means that any neighborhood $\prod_{i \in \N} U_i$ of $x$ intersects $\R^\infty$, thus all but finitely many $U_i$ must contain zero. Consider the neighborhood 
    \begin{align*}
        &V = \prod_{i \in \N} V_i \\
        &V_i = \begin{cases}
        (0, x_i + 1) & x_i > 0 \\
        (x_i-1, 0) & x_i < 0 \\
        \R & x_i = 0.
        \end{cases}
    \end{align*}
    Clearly we have $x \in V$, so there are only finitely many $V_i$ that do not contain zero, thus $V$ is eventually all of $\R$, but, by the construction of $V$, this can only happen if $x$ is eventually zero. Thus $x \in \R^\infty$, and $\closure{\R^\infty} = \R^\infty$ in the box topology.
    
    Next we show that $\closure{\R^\infty} = \R^\omega$ in the product topology. Let $x \in \R^\omega$ be arbitrary and let $U = \prod_{i \in \N} U_i$ be a neighborhood of $x$. Since $U$ is open in the product topology, every $U_i$ must be all of $\R$ whenever $i \geq I$ for some $I \in \N$. Thus, we have $y = (x_1, \dots, x_{I-1}, 0, 0, 0, \dots) \in \R^\infty$, and $y \in U$. Therefore $U \cap \R^\infty \neq \emptyset$, as we wanted to show.
    
\end{exc}

\end{document}